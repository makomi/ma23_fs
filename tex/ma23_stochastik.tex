%    Copyright (c)  2008, 2010, 2012, 2016  Matthias Kolja Miehl <miehl@w3hs.net>
%    This work is licensed under a Creative Commons Attribution-ShareAlike 4.0 International License (CC BY-SA 4.0).
%     - Attribution
%     - Share Alike
%
%    http://creativecommons.org/licenses/by-sa/4.0/
%
%    This license allows you to remix, tweak, and build upon the material,
%    as long as you credit the copyright holder and license your new
%    creations under the identical terms.


%    Motivation
%    Durch die Veröffenltichung der Formelsammlung unter der 'CC BY-SA 4.0' soll erreicht werden,
%    dass die Quelltexte aller Dokumente, die auf diesem Quelltext basieren, wiederum für jeden zugänglich sind.
%    Somit soll jeder von der Arbeit der anderen profitieren können und selbst die Möglichkeit haben andere
%    von seiner Arbeit profitieren zu lassen.
%    Nichts ist ärgerlicher als eine bereits getane Arbeit nur deshalb nocheinmal erledigen zu müssen,
%    weil man die ursprüngliche nicht für die eigenen Zwecke nutzen (z.B. für sich anpassen) darf.


%    Tipp zum Erstellen einer neuen Dokumentversion
%    Um die Stellen im Quelltext zu finden, die für eine neue Dokumentversion angepasst werden müssen,
%    einfach nach 'ZU ERLEDIGEN BEI NEUER DOKUMENTVERSION' suchen. Alle Stellen sind auf diese Weise markiert.


\documentclass[a4paper,10pt,titlepage]{scrartcl}

% META INFO %
\newcommand{\thisdocTITLE}{Ma23 Formelsammlung Stochastik}
\newcommand{\projectURL}{https://github.com/makomi/ma23\_fs/}
% ZU ERLEDIGEN BEI NEUER DOKUMENTVERSION:
% Datum und Version aktualisieren
\newcommand{\thisdocDATE}{2016-01-18 v1.0.3}
% eventuell neuer Herausgeber
\newcommand{\thisdocSUBJECT}{\projectURL} % Herausgeber:
% Neue Autoren anfügen (jeweils durch ein Komma getrennt)
\newcommand{\myNAME}{Matthias Kolja Miehl}
\newcommand{\myEMAIL}{miehl@w3hs.net}
\newcommand{\thisdocAUTHOR}{\myNAME, \myEMAIL}

% linespacing
\usepackage{setspace}
\spacing{0.8}
%\linespread{0.7}

% MISC PACKAGES %
\usepackage[utf8]{inputenc}
\usepackage[ngerman]{babel}
\selectlanguage{ngerman}
\usepackage{vmargin}
\usepackage{lastpage}
\usepackage{url}
% MATH PACKAGES %
\usepackage{amsmath,amssymb,amsthm}


% Document Properties
\usepackage{hyperref}
\hypersetup{%
            pdftitle        ={\thisdocTITLE}
           ,pdfauthor       ={\thisdocAUTHOR}
           ,pdfsubject      ={\thisdocSUBJECT}
           ,pdfkeywords     ={\thisdocDATE}
%           ,pdfcreator      ={}
%
           ,colorlinks      =true                       % false: boxed links; true: colored links
%           ,linkbordercolor ={1 1 1}                    % {r g b} (e.g. white: {1 1 1})
%           ,urlbordercolor  ={1 1 1}                    %
%           ,filebordercolor ={1 1 1}                    %
%           ,citebordercolor ={1 1 1}                    %
           ,linkcolor       =black                      % color of internal links        (e.g. red)
           ,urlcolor        =black                      % color of external links        (e.g. cyan)
           ,filecolor       =black                      % color of file links            (e.g. magenta)
           ,citecolor       =black                      % color of links to bibliography (e.g. green)
%
%           ,pdfnewwindow    =true                       % open links in new window?
%           ,pdfstartview    =FitBH                      % FitBH: fit width of page to the window
%           ,pdffitwindow    =true                       % page fit to window when opened
%           ,pdftoolbar      =true                       % show Acrobat’s toolbar?
%           ,pdfmenubar      =true                       % show Acrobat’s menu?
%           ,bookmarks       =false                      % show bookmarks bar? (SUGI style guide)
%           ,unicode         =true                       % false: non-Latin characters in Acrobat’s bookmarks
}

% Maximum margins for 'HP Deskjet 930c'
\setmargins{11px}{11px}       % linker          & oberer Rand
           {20.2cm}{28.2cm}   % Textbreite      & -höhe
           {14pt}{0cm}        % Kopfzeilenhoehe & -abstand
           {0pt}{0cm}         % \footheight     & Fusszeilenabstand

% Document Header
\usepackage{fancyhdr}
\pagestyle{fancy}
 \lhead{}
 \chead{\begin{scriptsize}{\thepage}\,/\,\pageref{LastPage}\end{scriptsize}}
 \rhead{\begin{scriptsize}\href{\projectURL}{\projectURL} \quad \href{mailto:\myEMAIL}{\myEMAIL} \quad \thisdocDATE\end{scriptsize}}
 \lfoot{}
 \cfoot{}
 \rfoot{}
\renewcommand{\headrulewidth}{0.0pt}
\renewcommand{\footrulewidth}{0.0pt}


% USED LABELs
%  
% USED COMMANDs
%  \binom{}{} \text{} \begin{Bmatrix}n\\r\end{Bmatrix}
%  \stackrel{!} \mathrel{\widehat{=}}
%  \mathrm{d}x

\begin{document}

\pagenumbering{alph}   % a, b, c, ...

\begin{titlepage}
  \vspace*{\fill}
  \begin{center}
    \huge
    Mathematik 2,\,3\\
    Formelsammlung zur Stochastik\\
    \vspace{1.5cm}
    \large
    \thisdocDATE
  \end{center}
  \vspace*{\fill}
  \begin{center}{\fontsize{9pt}{11pt}\selectfont
    % Creative Commons License Notice
    %
    This work is licensed under a\\[1em]
    \textbf{Creative Commons Attribution-ShareAlike 4.0 International License (CC BY-SA 4.0).\\[1em]}
    % More Information:
    \url{http://creativecommons.org/licenses/by-sa/4.0/}\\
  }
  \vspace*{\fill}
% ZU ERLEDIGEN BEI NEUER DOKUMENTVERSION:
% '-- Autor -- \medskip\\' auskommentieren und die darauf folgende Zeile einkommentieren; für jeden Autor eine eigene Zeile anfügen
%
  -- Autor -- \medskip\\
%  -- Autoren -- \medskip\\
  \myNAME, \href{mailto:\myEMAIL}{\myEMAIL}\smallskip\\
  \url{\projectURL}
% Vorlagen für weitere Zeilen:
%  Autor1 der modifizierten Version \\
%  Autor2 der modifizierten Version \\
%  Autor3 der Version, die auf der modifizierten Version aufbaut\\
%  ...
  \vspace*{1.4cm}
  \end{center}
\end{titlepage}

\newpage

\pagenumbering{arabic}   % 1, 2, 3, ...
\setcounter{page}{1}


\section*{Kombinatorik}
\label{sec:kombinatorik}

\begin{tabular}{lr}
\parbox{13cm}{%
%\noindent {\fontsize{10pt}{0pt} \bfseries Stichproben vom Umfang n aus Grundgesamtheit mit r Elementen}\\
\noindent {\fontsize{10pt}{0pt} \bfseries n-Stichproben aus r-Grundgesamtheit}
\\
\begin{tabular}{c|c|c|l}
   \parbox{3.4cm}{{\fontsize{10pt}{0pt}\selectfont Anz.\,d.\,Abb.}{\fontsize{8pt}{0pt} $f:N\to R$}}
 & \parbox{2.6cm}{{\fontsize{10pt}{0pt}\selectfont N unterscheidbar\\R unterscheidbar}}
 & \parbox{3.5cm}{{\fontsize{10pt}{0pt}\selectfont N \underline{nicht} unterscheidbar\\R unterscheidbar}}
 & 
 \\
\cline{1-3}
   \parbox{3.8cm}{{\fontsize{8pt}{0pt}\bfseries\selectfont beliebige Anz.}\\ von R Elementen}
 & $r^n$
 & $\binom{r+n-1}{n}$
 & \parbox{1.5cm}{{\fontsize{9pt}{0pt}\selectfont mit\dots}}
 \\
\cline{1-4}
   \parbox{3.8cm}{{\fontsize{8pt}{0pt}\bfseries\selectfont injektiv} (linkseindeutig): {\fontsize{10pt}{0pt}\selectfont \\max.\,ein Element aus R}}
 & $\frac{r!}{\left(r-n\right)!}=r^{\left(n\right)}$
 & $\frac{r^{(n)}}{n!}=\binom{r}{n}$ \, {\fontsize{7pt}{0pt}\selectfont $(|N|\leq |R|)$}
 & \parbox{1.5cm}{{\fontsize{9pt}{0pt}\selectfont ohne\dots}}
 \\
\cline{1-4}
   \parbox{3.8cm}{{\fontsize{8pt}{0pt}\bfseries\selectfont surjektiv} (rechtstotal): {\fontsize{10pt}{0pt}\selectfont \\mind.\,ein Element aus R}}
 & $\text{}$
 & $S_{(n,r)}
%=S_{n}^{(r)}
=\begin{Bmatrix}n\\r\end{Bmatrix}$ \, {\fontsize{7pt}{0pt}\selectfont $(|N|\geq |R|)$}
 & \parbox{1.5cm}{{\fontsize{9pt}{0pt}\selectfont ohne\dots}}
 \\
\cline{1-4}
   \parbox{3.8cm}{{\fontsize{8pt}{0pt}\bfseries\selectfont bijektiv} {\fontsize{8pt}{0pt}$(|R|=|N|)$}: {\fontsize{10pt}{0pt}\selectfont genau\\ein Element aus R und N}}
 & \parbox{1.5cm}{%
    $n^{(n)}=n!$\smallskip\\
    $\frac{n!}{n_1!\,n_2!\,\dots\,n_k!}$
   }
 & \parbox{3.5cm}{%
   {\fontsize{7pt}{0pt}\selectfont alle Elemente verschieden}\\
   {\fontsize{7pt}{0pt}\selectfont je $n_1,\,n_2,\,\dots,\,n_k$\,Elemente gleich\,(Anz.\,d.\,Klassen k\,$>$\,1)}
   }
 & \parbox{1.5cm}{{\fontsize{9pt}{0pt}\selectfont ohne\dots}}\\
\cline{1-3}

 & \parbox{1.2cm}{Variation\\geordnet}
 & \parbox{1.6cm}{Kombination\\ungeordnet}
 & \parbox{1.5cm}{{\fontsize{9pt}{0pt}\selectfont Wdhlg/\\Rücklegen}}
 \\
\end{tabular} 
}
&
\parbox{7cm}{%
{\fontsize{8pt}{10pt}\selectfont
\begin{tabular}{c|lllllllll}
\parbox{0.1cm}{n} & \parbox{0.01cm}{0} & \parbox{0.01cm}{1} & \parbox{0.01cm}{2} & \parbox{0.01cm}{3} & \parbox{0.01cm}{4} & \parbox{0.15cm}{5} & \parbox{0.15cm}{6} & \parbox{0.22cm}{7} & \parbox{0.4cm}{8}\\
\hline
\parbox{0.2cm}{$D_n$} & \parbox{0.01cm}{1} & \parbox{0.01cm}{0} & \parbox{0.01cm}{1} & \parbox{0.01cm}{2} & \parbox{0.01cm}{9} & \parbox{0.15cm}{44} & \parbox{0.15cm}{265} & \parbox{0.22cm}{1854} & \parbox{0.3cm}{14833}\\

\parbox{0.1cm}{n!} & \parbox{0.01cm}{1} & \parbox{0.01cm}{1} & \parbox{0.01cm}{2} & \parbox{0.01cm}{6} & \parbox{0.01cm}{24} & \parbox{0.15cm}{120} & \parbox{0.15cm}{720} & \parbox{0.22cm}{5040} & \parbox{0.3cm}{40320}\\
\end{tabular}
}\medskip\\
{\fontsize{8pt}{10pt}\selectfont
\begin{tabular}{c|cccccccccc}
\parbox{0.55cm}{$S_{(n,r)}$} & \parbox{0.01cm}{0} & \parbox{0.01cm}{1} & \parbox{0.01cm}{2} & \parbox{0.01cm}{3} & \parbox{0.01cm}{4} & \parbox{0.01cm}{5} & \parbox{0.01cm}{6} & \parbox{0.01cm}{7} & \parbox{0.01cm}{8} & {\bfseries [$r$]}\\
\hline
0 & \parbox{0.1cm}{1} & \parbox{0.01cm}{} & \parbox{0.01cm}{} & \parbox{0.01cm}{} & \parbox{0.01cm}{} & \parbox{0.01cm}{} & \parbox{0.01cm}{} & \parbox{0.01cm}{} & \parbox{0.01cm}{} & \\
1 & \parbox{0.1cm}{} & \parbox{0.01cm}{1} & \parbox{0.01cm}{} & \parbox{0.01cm}{} & \parbox{0.01cm}{} & \parbox{0.01cm}{} & \parbox{0.01cm}{} & \parbox{0.01cm}{} & \parbox{0.01cm}{} & \\
2 & \parbox{0.1cm}{} & \parbox{0.01cm}{1} & \parbox{0.01cm}{1} & \parbox{0.01cm}{} & \parbox{0.01cm}{} & \parbox{0.01cm}{} & \parbox{0.01cm}{} & \parbox{0.01cm}{} & \parbox{0.01cm}{} & \\
3 & \parbox{0.1cm}{} & \parbox{0.01cm}{1} & \parbox{0.01cm}{3} & \parbox{0.01cm}{1} & \parbox{0.01cm}{} & \parbox{0.01cm}{} & \parbox{0.01cm}{} & \parbox{0.01cm}{} & \parbox{0.01cm}{} & \\
4 & \parbox{0.1cm}{} & \parbox{0.01cm}{1} & \parbox{0.01cm}{7} & \parbox{0.01cm}{6} & \parbox{0.01cm}{1} & \parbox{0.01cm}{} & \parbox{0.01cm}{} & \parbox{0.01cm}{} & \parbox{0.01cm}{} & \\
5 & \parbox{0.1cm}{} & \parbox{0.01cm}{1} & \parbox{0.08cm}{15} & \parbox{0.08cm}{25} & \parbox{0.08cm}{10} & \parbox{0.01cm}{1} & \parbox{0.01cm}{} & \parbox{0.01cm}{} & \parbox{0.01cm}{} & \\
6 & \parbox{0.1cm}{} & \parbox{0.01cm}{1} & \parbox{0.08cm}{31} & \parbox{0.08cm}{90} & \parbox{0.08cm}{65} & \parbox{0.08cm}{15} & \parbox{0.01cm}{1} & \parbox{0.01cm}{} & \parbox{0.01cm}{} & \\
7 & \parbox{0.1cm}{} & \parbox{0.01cm}{1} & \parbox{0.08cm}{63} & \parbox{0.08cm}{301} & \parbox{0.08cm}{350} & \parbox{0.08cm}{140} & \parbox{0.01cm}{21} & \parbox{0.01cm}{1} & \parbox{0.01cm}{} & \\
8 & \parbox{0.1cm}{} & \parbox{0.01cm}{1} & \parbox{0.1cm}{127} & \parbox{0.1cm}{966} & \parbox{0.3cm}{1701} & \parbox{0.3cm}{1050} & \parbox{0.3cm}{266} & \parbox{0.08cm}{28} & \parbox{0.01cm}{1} & \\
{\bfseries [$n$]} & & & & & & & & & \\
\end{tabular}
}
}\\
\end{tabular}
\medskip\\
Anz.\,geordneter Zahlenpartitionen: $\binom{n-1}{r-1}-\binom{r}{1}\cdot\binom{n-1-m}{r-1}+\binom{r}{2}\cdot\binom{n-1-2m}{r-1}-\cdots$ \quad ($m$ = max.\,Größe einer Zahlenpartition)\\
\noindent Anz.\,surjektiver Abb.: $S_{(n,r)}\cdot k^{(r)}$; $k=r$ mit $r$=Anz.\,d.\,Halte, $k$=Anz.\,d.\,Etagen, $n$=Anz.\,d.\,Leute; \emph{Sonst:} $k\neq r$ (\emph{nicht} rechtstotal)\\
\noindent Anz. d. $n$-Permutationen einer $r$-Menge: $n!\binom{r}{n}$

\section*{Stochastik}
\label{sec:stochastik}
\begin{tabular}{l|l}
  \parbox{4.5cm}{%
   {\fontsize{10pt}{0pt} \bfseries Klassischer W.-Raum}\\
   {\fontsize{8pt}{0pt}\selectfont (alle\,Erg.\,gleichw.,\,Laplace-Münze)}
  }
& $P(A)=\frac{g}{m}=g\cdot\frac{1}{m}=\frac{|A|}{|\Omega|}\hspace{0.2cm}=\frac{\text{Anz.\,günstiger Elementarereignisse}}{\text{Anz.\,möglicher Ergebnisse}}=\frac{1}{\text{Anz.}}$ \,; Anz.\,mögl.\,Ereignisse: $2^{|\Omega|}$
\\

  {\fontsize{10pt}{0pt}\bfseries Bedingte W.}
& $P(A|B)=\frac{P(A\cdot B)}{P(B)}                        \hspace{1.28cm}=\frac{\text{W. des günstigsten Pfades}}{\Sigma \text{ d.\,W. aller Pfade, die zu }B\text{ führen}}$
\\

  \parbox{4.5cm}{%
   {\fontsize{10pt}{0pt} \bfseries Zufallsexperiment}\\
   {\fontsize{8pt}{0pt}\selectfont (Münzen,\,Würfel,\,Ziehen\,v.\,Kugeln)}
  }
& $P(A_i|B)=\frac{P(A_i)\cdot P(B|A_i)}{P(B)}             \hspace{0.21cm}=\frac{\text{W.\,des günstigen Pfades}}{\Sigma\text{ d.\,W.\,aller Pfade}}$ \hspace{2.6cm} (Formel v.\,Bayes)
\\
\end{tabular}


\subsection*{Wahrscheinlichkeitsverteilungen}
\label{sec:wahrscheinlichkeitsverteilungen}
%{\fontsize{8pt}{0pt}\selectfont W.\,f.\,einen Stichprobenvektor}\smallskip\\
\begin{tabular}{c|l|l}
		& kleines N	& großes N $(n\to\infty)$\\ \hline
   \raisebox{-4ex}{\parbox{1.6cm}{mit\\Rücklegen}}
 & \parbox[t]{6.0cm}{%
     {\fontsize{10pt}{0pt} \bfseries Binomialverteilung} ($n\,p\leq10$) 
     \begin{tabbing}$P(X=x)=f(x/n;p)$\=$=\binom{n}{x}\,p^{x}\,(1-p)^{n-x}$\\
       Sonderfall: \>$=\binom{n}{x}\,0.5^{n}$
     \end{tabbing}
   } 
 & \parbox[t]{10.85cm}{%
     {\fontsize{10pt}{0pt} \bfseries Poissonverteilung} $\binom{p\,\to\,0}{n\to\infty}$ ($n\geq1500\,p$) \medskip \\
     $f(x/\lambda)=\frac{\lambda^x}{x!}\,e^{-\lambda}=\text{rel.\,W.}=\frac{n_v}{n}$ 	\hspace{3mm} $\lambda=n\,p=\mu=Var(x)=\frac{\text{Anz.\,Vork.}}{\text{Anz.\,Zeitint.}}$\\
     $P(X_t=x)=\frac{(\lambda\,t)^x}{x!}\,e^{-\lambda\,t}$	 			\hspace{1.53cm} (Poisson-Prozess)
   }
\\ \hline

   \raisebox{-2.5ex}{\parbox{1.6cm}{ohne\\Rücklegen}}
 & \parbox[t]{6.0cm}{%
     {\fontsize{10pt}{0pt} \bfseries Hypergeometrische Verteilung}\\
     $f(x/N,N_1,n)=\frac{\binom{N_1}{x}\binom{N-N_1}{n-x}}{\binom{N}{n}}$
   }
 & \parbox[t]{10cm}{%
     {\fontsize{10pt}{0pt} \bfseries Binomialverteilung}\\
     $\lim \limits_{N\to\infty}\,f(x/N,N_1,n)=\binom{n}{x}\,p^x\,(1-p)^{n-x}$\,; \space\space\space\space\space\space $\frac{N_1}{N}=p=const.$
   }
\end{tabular}

\subsubsection*{Mehrdimensionale diskrete W.-Verteilung}
\label{sec:mehrdimensionale_diskrete_w}
\begin{tabular}{c|c|l}
  \parbox{1.8cm}{mit\\Rücklegen}
& \parbox{12.7cm}{%
    $P(X=x_1,X=x_2,\dots,X=x_k)=\frac{n!}{x_1!\,x_2!\,\dots\,x_k!}\left(\frac{N_1}{N}\right)^{x_1}\cdot\left(\frac{N_2}{N}\right)^{x_2}\cdot\,\dots\,\cdot\left(\frac{N_k}{N}\right)^{x_k}$
  }
& \parbox{3.7cm}{{\fontsize{10pt}{0pt} \bfseries Polynomialverteilung}\\Verallg. \\Binomialverteilung}
\\ \hline

  \parbox{1.8cm}{ohne\\Rücklegen}
& \parbox{12.7cm}{%
    $P(X=x_1,X=x_2,\dots,X=x_k)=\frac{\binom{N_1}{x_1}\,\binom{N_2}{x_2}\dots\binom{N_k}{x_k}}{\binom{N}{n}}$ \quad $n=\sum \limits_{k} x_k$ \; $N=\sum \limits_{k} N_k$
  }
& \parbox{3cm}{{\fontsize{10pt}{0pt} \bfseries Verallg.\\Hyperg. Vert.}}
\\
\end{tabular} 

\subsubsection*{Normalverteilung}
\label{sec:normalverteilung}
\begin{tabular}{l|ll}
  $f(x/\mu;\sigma^2)=\frac{1}{\sigma\sqrt{2\pi}}\cdot e^{-\frac{(x-\mu)^2}{2\sigma^2}}$
& $P(X\leq x)=F(x)=P(U\leq u)=\Phi(u)$
& $P(a\leq X\leq b)=\int_{a}^{b}f(x/\mu;\sigma^2)\,\mathrm{d}x=F(b)-F(a)$
\\

$F(x/\mu;\sigma^2)=\int_{-\infty}^{x}f(t/\mu;\sigma^2)\,\mathrm{d}t$
% FIX: Das folgende muss - laut Papula - folgendermaßen heißen:
%      $P(X>x)=P(U>u)=1-\Phi(u)$
%      Das gleiche gilt für die entsprechende Gleichung der Standardnormalverteilung - > REF-01.0
% REF-01.1
& $P(X\geq x)=P(U\geq u)=1-\Phi(u)$
& $P(x_1\leq X\leq x_2)=P(u_1\leq U\leq u_2)=\Phi(u_2)-\Phi(u_1)$
\\

  
% NOTE: Folgendes gilt für symmetrische Intervalle
& \multicolumn{2}{c}{W.\,dass sich $X$ weniger als $k\sigma$ von $\mu$ unterscheidet: $P(|X-\mu|\leq k\sigma)=P(|U|\leq k)=2\Phi(k)-1$}
\\
\end{tabular} 


\subsubsection*{Standardnormalverteilung \quad {\fontsize{8pt}{0pt}\mdseries\selectfont ($\Phi(-u)=1-\Phi(u)$)}}
\label{sec:standardnormalverteilung}
\begin{tabular}{l|l|ll}
  $\varphi(u)=f(u/0;1)=\frac{1}{\sqrt{2\pi}}\cdot e^{-\frac{u^2}{2}}$
& $U=\frac{X-\mu}{\sigma}$
& $P(U\leq c)=\Phi(c)$
& $P(a\leq U\leq b)=\Phi(b)-\Phi(a)$
\\

  $\Phi(u)=F(u/0;1)=\frac{1}{\sqrt{2\pi}}\int_{-\infty}^{u}e^{-\frac{t^2}{2}}\,\mathrm{d}t$
% NOTE: Die folgende Zeile ist eine spezielle Form der 'standardisierten Zufallsvariablen U', 
%       die bei Parameterschätzungen bzw. -tests in dieser Form benötitgt wird.
%       (s. Papula: Band 3, Auflage 4, S. 513)
& $U=\frac{(X-\mu)\sqrt{n}}{\sigma}$
& $P(U\geq c)=1-P(U\leq c)$
& $P(-c\leq U\leq c)=P(|U|\leq c)=2\Phi(c)-1$
\\
\end{tabular}

\smallskip
\noindent{\fontsize{8pt}{0pt} \bfseries Quantile der Standardnormalverteilung} \quad {\fontsize{8pt}{0pt}\mdseries\selectfont ($-u_p=u_{1-p}$)}\\
\begin{tabular}{l|ll|l}
  {\fontsize{8pt}{0pt}\selectfont Vorgegebene W.: \qquad}$p$
& $P(U\leq c)=\Phi(c)=p$
& $\left[ \Phi(c)=p \hspace{7mm} \to c=u_p \right]$
& $P(U\leq u_p)=\Phi(u_p)=p$
\\

  {\fontsize{8pt}{0pt}\selectfont Zugehöriges Quantil: }$u_p$
% FIXME: Das folgende muss - laut Papula - folgendermaßen heißen:
%        $P(U>c)=1-P(U\leq c)=p$
%        Das gleiche gilt für die entsprechende Gleichung der Normalverteilung -> REF-01.1
% REF-01.0
& $P(U\geq c)=1-P(U\leq c)=p$
& $\left[ \Phi(c)=1-p \hspace{1mm} \to c=u_{1-p} \right]$
& $P(U\leq 1,645)=95\%$, \, $u_{0,95}=1,645$
\\
  
& $P(-c\leq U \leq c)=2\Phi(c)-1=p$
& $\left[ \Phi(c)=\frac{1+p}{2} \hspace{2.6mm} \to c=u_{(1+p)/2} \right]$
& $P(u_1\leq U\leq u_2)=\Phi(u_2)-\Phi(u_1)$
\\
\end{tabular}
\smallskip \\

\noindent\begin{tabular}{l}
$F(X)=\int_{a}^{b}f(x)\,\mathrm{d}x\stackrel{!}=1$
\quad $\mathrm{E}(X)=\mu=\int_{-\infty}^{\infty}x\,f(x)\,\mathrm{d}x=\sum\limits_{i}x_i f(x_i)$
\quad $\mathrm{E}(X^2)=\int_{-\infty}^{\infty}x^2\,f(x^2)\,\mathrm{d}x$
\\ \hline
$\sigma^2=\mathrm{Var}(X)=\mathrm{E}(X^2)-(\mathrm{E}(X))^2=\int_{-\infty}^{\infty}(x-\mu)^2f(x)\,\mathrm{d}x$
\quad $\mathrm{Var}(aX+b)=a^2\,\mathrm{Var}(X)$
\quad $\mathrm{Cov}(X,Y)=\mathrm{E}(X\cdot Y)-\mathrm{E}(X)\cdot \mathrm{E}(Y)$
\\
\end{tabular}

% NOTE: Ich hatte ursprünglich vor die beurteilende Statistik mit in diese FS aufzunehmen.
%       Durch die zusätzliche Verwendung der Papula Formelsammlung, habe ich mich aus Zeitgründen jedoch letztendlich dagegen entschieden.
%       Somit verbleibt dieser Stumpf als Anregung.
%\section*{Beurteilende Statistik}
%\label{sec:beurteilende_statistik}
%\bigskip\\
%\noindent\begin{center}\rule{20.2cm}{0.2mm}\end{center}
\noindent {\fontsize{9pt}{0pt}\bfseries Zum Binomialkoeffizienten} \smallskip \\
{\fontsize{9pt}{0pt}\selectfont
 $\binom{r}{n}=\frac{r!}{n!\,(r-n)!}=\binom{r}{r-n}$ \;$|\!|$\;
 $\binom{r}{0}=\binom{r}{r}=1$ \quad
 $\binom{r}{1}=\binom{r}{r-1}=r$ \; $\binom{r}{r+1}=0$ \;$|\!|$\; $\binom{r}{n}=\binom{r-1}{n-1}+\binom{r-1}{n}$ \quad
 Sym.: $\binom{r}{n}=\binom{r}{r-n}$ \quad Sum.: $\binom{r}{n}+\binom{r}{n+1}=\binom{r+1}{n+1}$
}
\smallskip\\
{\fontsize{9pt}{0pt}\selectfont
 $\sum\limits_{n=0}^{r}\cdot\binom{r}{n}=2^r$ \quad $n\cdot \binom{r}{n}=r\cdot \binom{r-1}{n-1}$
 \quad $\sum\limits_{k=0}^{n}\binom{r}{k}\binom{S}{n-k}=\binom{r+S}{n}$ {\fontsize{8pt}{0pt}\selectfont (Vandermond'sche Id.)}\quad
 %\hspace{.7cm}$||$\hspace{0.7cm} \smallskip\\
 % Mit folgendem weis ich nichts anzufangen...
 $\sum\limits_{k=0}^{r}\binom{k}{n}=\binom{r+1}{n+1}$ \quad $\sum\limits_{k=0}^{r}\binom{n+k}{n}=\binom{r+n+1}{n+1}$ \quad $\sum\limits_{n=0}^{r}\binom{n+k}{n}=\binom{k+r+1}{r}$
}


\newpage

\section*{Notizen}
\label{sec:notizen}
\noindent
$r$-Partition: Zerlegung in $r$ (disjunkte) Einzelblöcke\\
$r$-Permutation: Permutation von $r$ Elementen \; (Permutation: Jede Anordnung von $n$ Elementen in einer bestimmten Reihenfolge)\\
\noindent\rule[2pt]{20.2cm}{0.1mm}

$r^n$\\
Anz.\,d.\,Mögl.\,$n$-Mal \emph{unter Beachtung der Reihenfolge} ein Element \emph{mit Rücklegen} aus einer Menge vom Umfang $r$ auszuwählen.\\
\emph{Bsp.:} Anz.\,d.\,mögl.\,Binärzahlen: $2^n$; $r=2$ Zustände $\{0,1\}$ aus denen $n$ mal gewählt wird ($n\mathrel{\widehat{=}}$ Länge bzw.\,Stellen der Zahl).\\
(Bei erster Stelle zwei Möglichkeiten sich zu entscheiden, bei zweiter wieder zwei ($2\cdot2$ Möglichkeiten) \dots)\\

$r^{(n)}$ \quad {\fontsize{8pt}{0pt}\selectfont (Faktorielle Potenz)}\\
Anz.\,d.\,Mögl.\,$n$-Mal \emph{unter Beachtung der Reihenfolge} ein Element \emph{ohne Rücklegen} aus einer $r$-Menge auszuwählen.\\
\emph{Bsp.:} Anz.\,d.\,Mögl.\,für die ersten $n$ Zieleinläufe von $r$ Läufern.\\
\emph{Sonderfall:} $r=n$ $\,\to\,$ $n^{(n)}=n!$ : Anz.\,d.\,Mögl.\,für verschiedene Zieleinläufe von $n$ Läufern, wenn alle ins Ziel kommen ($n$ Zieleinläufe).
\noindent\rule[3pt]{20.2cm}{0.1mm}

$\binom{r}{n}=\frac{r^{(n)}}{n!}$ \quad {\fontsize{8pt}{0pt}\selectfont$(|N|\leq|R|)$} \quad {\fontsize{8pt}{0pt}\selectfont (Binomialkoeffizient)}\\
Anz.\,d.\,Mögl.\,\emph{ohne Beachtung der Reihenfolge} und \emph{ohne Rücklegen} $n$ Elemente aus $r$ auszuwählen. {\fontsize{8pt}{0pt}\selectfont Anz.\,d.\,$k$-Teilmengen einer $n$-Menge}\\
$n \mathrel{\widehat{=}}$ Anz.\,d.\,Mögl.\,(aus R) auszuwählen \quad $r \mathrel{\widehat{=}}$ Anz.\,der Elemente aus denen ich auswählen kann\\
\emph{Bsp.:} Anz.\,d.\,Mögl.\,$n$ Kugeln auf $r$ Zellen zu verteilen. (jede Zelle max.\,nur eine Kugel, Reihenfolge egal -- Ergeb.\,zählt) \; $|\cdot|\;\;\,|\cdot|\cdot|\;\;\,|$\\
\emph{Bsp.:} \{Spatzen\}\,$\to$\,\{Stromleitungen\}, \{Kugeln\}\,$\to$\,\{Zellen\}, \{Plätze im Straus\}\,$\to$\,\{Blumentypen\}\\

$\binom{r}{2}$ \quad {\fontsize{8pt}{0pt}\selectfont (Permutation + Vertauschungen)}\\
Anz.\,d.\,möglichen Vertauschungen bei $r$ Elementen. Jede $r$-Permutation kann mit max. $r-1$ Vertauschungen realisiert\nolinebreak[5] werden.\\

$\binom{r+n-1}{n}$\\
Anz.\,d.\,Mögl.\,\emph{ohne Beachtung der Reihenfolge} und \emph{mit Wiederholung} $n$-Elemente aus $r$ Elementen auszuwählen.\\
\emph{Bsp.:} Das gleiche wie der Binomialkoeffizient, doch nun mehrere Kugeln in einer Zelle erlaubt.\quad $|\cdot|\;\;\;|:\,:|:\cdot|\;\;\;|$ {\fontsize{7pt}{0pt}\selectfont $\mathrel{\widehat{=}} 011000010001$}\\

$\binom{r+n-1}{r-1}$\\
\emph{Bsp.:} Anz.\,d.\,Mögl.\,$4$ Einsen auf $10$ Stellen zu verteilen. $(n+r-1)$-stellige Dualzahl mit $(r-1)$ Einsen/Trennwände, $r$ Kästchen\\

$P_{n,k}=\binom{n-1}{r-1}$\\
Anz.\,d.\,geordneten Zahlenpartitionen: Anz.\,d.\,Mögl.\,$n$ aus $r$ Partitionen\,/\,Zahlen zusammenzusetzen (geordnet).\\
\emph{Bsp.:} Anz.\,d.\,Mögl.\,die Zahl $n$ in $r$ Zahlenpartitionen einzuteilen: $\binom{4-1}{2-1}=3$ und zwar $(3+1)=(1+3)=(2+2)$\medskip\\
\noindent
Max. Größe einer Partition\,/\,Zahl auf $m$ beschränken: $\binom{n-1}{r-1}-\binom{r}{1}\cdot\binom{n-1-m}{r-1}+\binom{r}{2}\cdot\binom{n-1-2m}{r-1}-\cdots$\\
\emph{Bsp.:} Anz.\,d.\,Mögl.\,mit $3$ Würfeln die Augenzahl $8$ zu werfen: $n=8$, $r=3$, $m=6$\\
\emph{Hier:} Beschränkung der max.\,Größe einer Zahlenpartition auf die max.\,Augenzahl eines Würfels ($m=6$).\\
\emph{Bsp.:} Parlament: $n=23$ Sitze; $r=3$ Parteien; max.\,Größe einer Partei ist $m=11$, um eine absolute Mehrheit zu verhindern.
\noindent\rule[3pt]{20.2cm}{0.1mm}

$n^{(n)}=n!$\\
Anz.\,d.\,Mögl.\,$n$ Elemente \emph{unter Beachtung der Reihenfolge} und \emph{ohne Rücklegen} in verschiedene Reihenfolgen zu bringen.\\
\emph{Bsp.:} Anz.\,d.\,Mögl.\,für verschiedene Zieleinläufe von $n$ Läufern, wenn alle ins Ziel kommen ($n$ Zieleinläufe).\\

$\frac{n!}{n_1!n_2!\dots n_k!}$ \quad {\fontsize{8pt}{0pt}\selectfont (Multinomialkoeffizient)}\\
Anz.\,d.\,Mögl.\,$n$ Elemente bestehend aus $k>1$ Klassen in unterschiedliche Reihenfolgen zu bringen. Verallg.\,Binomialkoeff.\,für $k>1$\\
\emph{Bsp.:} Anz.\,d.\,Mögl.\,aus ABRAKADABRA durch Vertauschen der Buchstaben neue Wörter zu bilden: $\frac{11!}{5!\,2!\,2!\,1!\,1!}$\\
\emph{Bsp.:} Anz.\,d.\,Mögl.\,$n_1$ rote und $n_2$ schwarze Kugeln auf $n=n_1+n_2+n_3$ Zellen zu verteilen mit $n_3$ leeren Zellen.\\

$D_n=n!\sum\limits_{k=0}^{n}\frac{(-1)^k}{k!} \qquad \lim\limits_{n\to\infty}D_n=\frac{n!}{e}$\\
{\fontsize{8pt}{0pt}\selectfont Fixpktfreie Permutationen: Anz.\,d.\,Mögl.\,$n$ Objekte $n$ anderen Objekten zuzuordnen, ohne dass die Nummern von zwei Objekten übereinstimmen.}\\
\emph{Bsp.:} Anz.\,d.\,Mögl.\,für Zieleinläufe, bei denen der Läufer mit der Nr.\,1 nicht an erster Stelle, der Läufer mit der Nr.\,2 nicht an zweiter Stelle usw.\,einläuft\,/\,einlaufen soll, darf.\\

$n!\binom{r}{n}$\\
Anz.\,d.\,$n$-Permutationen einer $r$-Menge: Anz.\,d.\,Permutationen von $n$ Elementen einer Menge vom Umfang $r$\\
% FIX: Beispiel hinzufügen
\emph{Bsp.:} \\

$S_{(n,r)}=\frac{1}{r!}\sum\limits_{i=0}^{r}\left[ (-1)^{r-i}\binom{r}{i}i^n \right]=\begin{Bmatrix}n\\r\end{Bmatrix}$ \quad {\fontsize{8pt}{0pt}\selectfont$(|N|\geq|R|)$} \quad {\fontsize{8pt}{0pt}\selectfont (Mengenpartition, Stirlingzahl 2.\,Art)}\\
Anz.\,d.\,$r$-Partitionen einer $n$-Menge (ohne Rücklegen, ungeordnet): Anz.\,d.\,Mögl.\,eine $n$-Menge in $r$ disjunkte Teilmengen einzuteilen.\\
Anz.\,d.\,Mögl.\,$n$ Objekte in Partitionen vom Umfang $n$ einzuteilen.\\
\emph{Bsp.:} $n$ Studenten sollen $r$ Arbeitsgruppen bilden.\\
\emph{Bsp.:} $\begin{Bmatrix}4\\2\end{Bmatrix}=7$ : \{A\,B\}\{C\,D\}, \{A\,C\}\{B\,D\}, \{A\,D\}\{C\,B\}, \{A\,B\,C\}\{D\}, \{A\,B\,D\}\{C\}, \{A\,C\,D\}\{B\}, \{B\,C\,D\}\{A\} \\
\begin{tabular}{llll}
  $\begin{Bmatrix}n\\r\end{Bmatrix}=\begin{Bmatrix}n-1\\r-1\end{Bmatrix}+r\begin{Bmatrix}n-1\\r\end{Bmatrix}$
& \quad Sonderwerte:
& $S_{(n,1)}=1$
& $S_{(n,2)}=2^{n-1}-1$
\\
  
& 
& $S_{(n,n)}=1$
& $S_{(n,n-1)}=\binom{n}{2}$
\\
\end{tabular}
\\

$S_{(n,r)}\cdot k^{(r)}$\\
\emph{Bsp.:} Anz.\,d.\,Mögl.\,für einen Fahrstuhl zu halten: $r$ = Anz.\,d.\,Halte, $k$ = Anz.\,d.\,Etagen, $n$ = Anz.\,d.\,Leute\\
\emph{$k=r$ :} Anz.\,surjektiver Abb. $f: N\to R$ \quad $|N|\geq|R|$ \quad (rechtstotal, jedes Element $\in R$ mind.\,1-mal)\\
\indent\emph{Bsp.:} Fahrstuhl hält auf allen Etagen\\
\emph{$k\neq r$ :} \emph{nicht} rechtstotal\\
\indent\emph{Bsp.:} Fahrstuhl hält \emph{nicht} auf allen Etagen\smallskip\\
($r^n$ Mögl.\,insg.)


\newpage

\begin{flushright}{\fontsize{8pt}{0pt}\selectfont Papula FS, S.\,367\,ff.}\end{flushright}

\section*{Axiome und Sätze}
\label{sec:axiome_und_saetze}

\subsection*{Poisson-Verteilung}
\label{sec:poisson-verteilung}
\begin{enumerate}
 \item Die W. für $x$ Signale in einem Zeitinterval der Länge $t$ hängt nur von $x$ und $t$ ab, nicht von der Lage des Zeitintervalls auf der Zeitachse.
 \item Die Anz. der Signale in disjunkten Zeitintervallen sind unabhängige Zufallsgrößen.
 \item Die W. für mehr als ein Signal in einem kleinen Zeitintervall der länge $\Delta t$ ist von kleinerer Größenordnung als $\Delta t$.
\end{enumerate}

\subsection*{Totale W., Formel von Bayes}
\label{sec:totale_w}
\begin{enumerate}
\item Multiplikationssatz: 
Die W. dafür, dass ein bestimmter Pfad durchlaufen wird, ist gleich dem Produkt der W. längs des Pfades.
\item Satz v.\,d.\,totalen W.: 
Die W. eines Ereignisses ist gleich der Summe der W. aller Pfade, die zu diesem Ereignis führen.
\end{enumerate}

%\vspace{.5cm}

\subsection*{Additionssatz}
\label{sec:additionssatz}
\begin{tabular}{ll}
disjunkt:       & $P(A_1+A_2)=P(A_1)+(A_2)$\\
nicht disjunkt: & $P(A_1+A_2)=P(A_1)+P(A_2)-P(A_1\cdot A_2)$\\
		& $P(A_1+A_2+A_3)=P(A_1)+P(A_2)+P(A_3)-P(A_1\cdot A_2)-P(A_1\cdot A_3)-P(A_2\cdot A_3)+P(A_1\cdot A_2\cdot A_3)$\\
\end{tabular}

\subsection*{Multiplikationssatz}
\label{sec:multiplikationssatz}
$P(A_1\cdot A_2)=P(A_1)\cdot P(A_2|A_1)$ \qquad $P(A_1\cdot A_2\cdot A_3)=P(A_1)\cdot P(A_2|A_1)\cdot P(A_3|A_1\cdot A_2)$

\subsection*{Faltungssatz}
\label{sec:faltungssatz}
$P(X_1+X_2=k)=\mathrm{e}^{-(\mu_1+\mu_2)}\cdot\frac{(\mu_1+\mu_2)^k}{k!}$ \quad wenn $X_1$ und $X_2$ poisson-verteilt mit $\mu_1$ und $\mu_2$\\
$\mathrm{E}(aX\pm bY)=a\mathrm{E}(X)\pm b\mathrm{E}(Y)$\\
$\mathrm{Var}(aX\pm bY)=a^2\mathrm{Var}(X)+b^2\mathrm{Var}(Y)\pm 2ab\,\mathrm{Cov}(X,Y)$

\section*{Rest}
\label{sec:rest}
\subsection*{Vergleiche}
\label{sec:vergleiche}
\emph{Binomialkoeffizient:} keine Elemente aus N, alles \emph{eine} Klasse\\
\emph{Multinomialkoeffizient:} mind.\,2 Klassen, Klassen mit bloß einem Element sind möglich\medskip\\
Klassenelemente \emph{nicht} unterscheidbar; Klassen untereinander unterscheidbar
\bigskip\\
$\binom{r}{n}$ : Anz.\,d.\,Mögl.\,$n$ Elemente auf $r$ Elemente zu verteilen (max.\,ein Element $\in$ N je Element $\in$ R)\\
$\begin{Bmatrix}n\\r\end{Bmatrix}$ : Anz.\,d.\,Mögl.\,eine Menge vom Umfang $n$ in disjunkte Teilmengen vom Umfang $r$ zusammenzufassen\\

\subsection*{Erklärung}
\label{sec:erklaerung}
\subsubsection*{Poisson-Verteilung}
\label{sec:poisson-verteilung_erklaerung}
\emph{Bsp.:} Huftritt\smallskip\\
\begin{tabular}{l||l|l|l|l|l|lll}
 $x_i$ & 0  & 1  & 2 & 3 & 4 & \dots & & Anz.\,d.\,toten Soldaten\\ \cline{0-6}
 $n_i$ & 15 & 11 & 6 & 2 & 0 & \dots & & Anz.\,d.\,Jahre, in denen entsprechend viele Soldaten durch Huftritt starben\\
\end{tabular}
\medskip\\
Die Anzahl der Jahre kann man sich auch als Anzahl von Zettelchen vorstellen, die mit der Jahreszahl bedruckt sind und in nummerierte Kästchen gelegt werden.\bigskip\\
Mittelwert: $\lambda=E(x)=\frac{\text{Anz.\,d.\,Vorkommnisse}}{\text{Anz.\,Zeitintervalle}}=\frac{15\cdot 0+11\cdot 1+6\cdot 2+2\cdot 3+0\cdot 4+\cdots}{5+4+2+0+1+\cdots}$
\smallskip\\
$f(x/\lambda)=$ Wahrscheinlichkeit für $x$ Tote in einem Zeitintervall (z.\,B.\,der Dauer eines Jahres)

\end{document}
