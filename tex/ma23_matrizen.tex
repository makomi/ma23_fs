%    Copyright (c)  2008, 2010, 2012, 2016  Matthias Kolja Miehl <miehl@w3hs.net>
%    This work is licensed under a Creative Commons Attribution-ShareAlike 4.0 International License (CC BY-SA 4.0).
%     - Attribution
%     - Share Alike
%
%    http://creativecommons.org/licenses/by-sa/4.0/
%
%    This license allows you to remix, tweak, and build upon the material,
%    as long as you credit the copyright holder and license your new
%    creations under the identical terms.


%    Motivation
%    Durch die Veröffenltichung der Formelsammlung unter der 'CC BY-SA 4.0' soll erreicht werden,
%    dass die Quelltexte aller Dokumente, die auf diesem Quelltext basieren, wiederum für jeden zugänglich sind.
%    Somit soll jeder von der Arbeit der anderen profitieren können und selbst die Möglichkeit haben andere
%    von seiner Arbeit profitieren zu lassen.
%    Nichts ist ärgerlicher als eine bereits getane Arbeit nur deshalb nocheinmal erledigen zu müssen,
%    weil man die ursprüngliche nicht für die eigenen Zwecke nutzen (z.B. für sich anpassen) darf.


%    Tipp zum Erstellen einer neuen Dokumentversion:
%    Um die Stellen im Quelltext zu finden, die für eine neue Dokumentversion angepasst werden müssen,
%    einfach nach 'ZU ERLEDIGEN BEI NEUER DOKUMENTVERSION' suchen. Alle Stellen sind auf diese Weise markiert.


\documentclass[a4paper,10pt,titlepage]{scrartcl}

% META INFO %
\newcommand{\thisdocTITLE}{Ma23 Formelsammlung Matrizen (\"{U}bersicht)}
\newcommand{\projectURL}{https://github.com/makomi/ma23\_fs/}
% ZU ERLEDIGEN BEI NEUER DOKUMENTVERSION:
% Datum und Version aktualisieren
\newcommand{\thisdocDATE}{2016-01-18 v1.0.3}
% eventuell neuer Herausgeber
\newcommand{\thisdocSUBJECT}{\projectURL} % Herausgeber:
% ZU ERLEDIGEN BEI NEUER DOKUMENTVERSION:
% Neue Autoren anfügen (jeweils durch ein Komma getrennt)
\newcommand{\myNAME}{Matthias Kolja Miehl}
\newcommand{\myEMAIL}{miehl@w3hs.net}
\newcommand{\thisdocAUTHOR}{\myNAME, \myEMAIL}

% linespacing
\usepackage{setspace}
\spacing{0.8}
%\linespread{0.7}

% MISC PACKAGES %
\usepackage[utf8]{inputenc}
\usepackage[ngerman]{babel}
\selectlanguage{ngerman}
\usepackage{vmargin}
\usepackage{url}
% MATH PACKAGES %
\usepackage{amsmath,amssymb,amsthm}


% Document Properties
\usepackage{hyperref}
\hypersetup{%
            pdftitle        ={\thisdocTITLE}
           ,pdfauthor       ={\thisdocAUTHOR}
           ,pdfsubject      ={\thisdocSUBJECT}
           ,pdfkeywords     ={\thisdocDATE}
%           ,pdfcreator      ={}
%
           ,colorlinks      =true                       % false: boxed links; true: colored links
%           ,linkbordercolor ={1 1 1}                    % {r g b} (e.g. white: {1 1 1})
%           ,urlbordercolor  ={1 1 1}                    %
%           ,filebordercolor ={1 1 1}                    %
%           ,citebordercolor ={1 1 1}                    %
           ,linkcolor       =black                      % color of internal links        (e.g. red)
           ,urlcolor        =black                      % color of external links        (e.g. cyan)
           ,filecolor       =black                      % color of file links            (e.g. magenta)
           ,citecolor       =black                      % color of links to bibliography (e.g. green)
%
%           ,pdfnewwindow    =true                       % open links in new window?
%           ,pdfstartview    =FitBH                      % FitBH: fit width of page to the window
%           ,pdffitwindow    =true                       % page fit to window when opened
%           ,pdftoolbar      =true                       % show Acrobat’s toolbar?
%           ,pdfmenubar      =true                       % show Acrobat’s menu?
%           ,bookmarks       =false                      % show bookmarks bar? (SUGI style guide)
%           ,unicode         =true                       % false: non-Latin characters in Acrobat’s bookmarks
}

% Maximum margins for 'HP Deskjet 930c'
\setmargins{11px}{11px}       % linker          & oberer Rand
           {20.2cm}{27.8cm}   % Textbreite      & -höhe
           {14pt}{0cm}        % Kopfzeilenhoehe & -abstand
           {0pt}{0cm}         % \footheight     & Fusszeilenabstand

% Document Header
\usepackage{fancyhdr}
\pagestyle{fancy}
 \lhead{}
 \chead{}
 \rhead{\begin{scriptsize}\href{\projectURL}{\projectURL} \quad \href{mailto:\myEMAIL}{\myEMAIL} \quad \thisdocDATE\end{scriptsize}}
 \lfoot{}
 \cfoot{}
 \rfoot{}
\renewcommand{\headrulewidth}{0.0pt}
\renewcommand{\footrulewidth}{0.0pt}


% USED LABELs
%  'FIX'
% USED COMMANDs
% \mathrel{\widehat{=}}
% \times

\begin{document}

\pagenumbering{alph}   % a, b, c, ...

\begin{titlepage}
  \vspace*{\fill}
  \begin{center}
    \huge
    Mathematik 2,\,3\\
    Übersicht zu Matrizen\\
    \vspace{1.5cm}
    \large
    \thisdocDATE
  \end{center}
  \vspace*{\fill}
  \begin{center}{\fontsize{9pt}{11pt}\selectfont
    % Creative Commons License Notice
    %
    This work is licensed under a\\[1em]
    \textbf{Creative Commons Attribution-ShareAlike 4.0 International License (CC BY-SA 4.0).\\[1em]}
    % More Information:
    \url{http://creativecommons.org/licenses/by-sa/4.0/}\\
  }
  \vspace*{\fill}
% ZU ERLEDIGEN BEI NEUER DOKUMENTVERSION:
% '-- Autor -- \medskip\\' auskommentieren und die darauf folgende Zeile einkommentieren; für jeden Autor eine eigene Zeile anfügen
%
  -- Autor -- \medskip\\
%  -- Autoren -- \medskip\\
  \myNAME, \href{mailto:\myEMAIL}{\myEMAIL}\smallskip\\
  \url{\projectURL}
% Vorlagen für weitere Zeilen:
%  Autor1 der modifizierten Version \\
%  Autor2 der modifizierten Version \\
%  Autor3 der Version, die auf der modifizierten Version aufbaut\\
%  ...
  \vspace*{1cm}
  \end{center}
\end{titlepage}

\newpage

\pagenumbering{arabic}   % 1, 2, 3, ...
\setcounter{page}{1}


\section*{Matrizen (Ergänzungen, Übersicht)}
\label{sec:matrizen}


\subsection*{Determinanten {\fontsize{9pt}{0pt}\selectfont\normalfont (Papula FS S.\,200--205)}}
\label{sec:determinanten}
\begin{tabular}{ll}
\emph{Verfahren:} 
& Zunächst alle Elemente bis auf eins in einer Zeile (oder Spalte) zu Null machen.
\\
 
& Anschließend die Determinante nach den Elementen dieser Zeile (oder Spalte) entwickeln. (Papula FS S.\,205)
\\ 
\end{tabular}\\
Matrix in Diagnonal- bzw.\,Dreiecksform: \quad Determinante $\mathrel{\widehat{=}}$ \emph{Produkt der Hauptdiagonalelemente}\\
Die Determinante ist gleich dem \emph{Produkt aller Eigenwerte}.

\subsection*{Eigenwerte und Eigenvektoren {\fontsize{9pt}{0pt}\selectfont\normalfont (Papula FS S.\,216\,ff.)}}
\label{sec:eigenwerte_und_eigenvektoren}
\emph{Verfahren:} $\text{det} (A-\lambda E)=p(\lambda)=0$\\
Eigenwerte sind nur dann vorhanden, wenn die Koeffizientendeterminante $\neq0$\\
Matrix in Diagnonal- bzw.\,Dreiecksform: \quad Eigenwerte $\mathrel{\widehat{=}}$ \emph{Hauptdiagonalelementen} \quad $\lambda_i=a_{ii}$ $(i=1, 2, \dots, n)$

\subsection*{Inverse {\fontsize{9pt}{0pt}\selectfont\normalfont (Papula FS S.\,196\,f.)}}
\label{sec:inverse}
\emph{Verfahren:} Gauß-Jordan-Verfahren\\
Die Inverse existiert nur für  \emph{quadratische} Matrizen, die zudem \emph{regulär} sind (Determinante $\neq0$).
\bigskip\\


\subsection*{Rang {\fontsize{9pt}{0pt}\selectfont\normalfont (Papula FS S.\,198)}}
\label{sec:rang}
\emph{Verfahren:} Trapezform\\
Für den Rang $\text{Rg}(A)$ einer $(m, n)$-Matrix A gilt: \quad $\text{Rg}(A)\leq m$ \quad (Rang \emph{höchstens gleich} der Anz.\,d.\,Zeilen der Matrix)

\subsection*{Allg.\,Lsg{\fontsize{9pt}{0pt}\selectfont\normalfont (Papula FS S.\,209\,f.)}}
\label{sec:allg_lsg}
\emph{Verfahren:} Gauß'scher Algorithmus
\bigskip\\


\subsection*{Produkt {\fontsize{9pt}{0pt}\selectfont\normalfont (Papula FS S.\,195)}}
\label{sec:produkt}
\emph{Verfahren:} Falk-Schema\\
\emph{Bedingung:} Das Produkt $A\cdot B$ ist nur möglich, wenn die \emph{Spaltenzahl} von A mit der \emph{Zeilenzahl} von B übereinstimmt.\\
\hspace*{2cm} $(2\times\text{{\bfseries\selectfont 3}})\cdot(\text{{\bfseries 3}}\times3)=(2\times3)$

\subsection*{Potenz}
\label{sec:potenz}
% FIX: Mit Inhalt füllen. (s. Ma23 Klausur Anfang SS08)



\end{document}
