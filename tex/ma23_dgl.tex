%    Copyright (c)  2008, 2010, 2012, 2016  Matthias Kolja Miehl <miehl@w3hs.net>
%    This work is licensed under a Creative Commons Attribution-ShareAlike 4.0 International License (CC BY-SA 4.0).
%     - Attribution
%     - Share Alike
%
%    http://creativecommons.org/licenses/by-sa/4.0/
%
%    This license allows you to remix, tweak, and build upon the material,
%    as long as you credit the copyright holder and license your new
%    creations under the identical terms.


%    Motivation
%    Durch die Veröffenltichung der Formelsammlung unter der 'CC BY-SA 4.0' soll erreicht werden,
%    dass die Quelltexte aller Dokumente, die auf diesem Quelltext basieren, wiederum für jeden zugänglich sind.
%    Somit soll jeder von der Arbeit der anderen profitieren können und selbst die Möglichkeit haben andere
%    von seiner Arbeit profitieren zu lassen.
%    Nichts ist ärgerlicher als eine bereits getane Arbeit nur deshalb nocheinmal erledigen zu müssen,
%    weil man die ursprüngliche nicht für die eigenen Zwecke nutzen (z.B. für sich anpassen) darf.


%    Literatur-Empfehlungen
%    1)  Gregor Oberholz - Differentialgleichungen für technische Berufe
%        2. Auflage 1989, ISBN 3-9801902-4-2, Verlag: 'Verlag Anita Oberholz', Gelsenkirchen
%        (Ausleihbar von der Stadtbibliothek Lübeck: http://www.luebeck.de/bewohner/beruf_arbeit/stadtbibliothek/)
%    2)  Lothar Papula - Mathematik für Ingenieure und Naturwissenschaftler - Band 2
%        9. Auflage 2000, ISBN 3-528-84237-7, Verlag: 'Vieweg', Braunschweig/Wiesbaden


%    Tipp zum Erstellen einer neuen Dokumentversion
%    Um die Stellen im Quelltext zu finden, die für eine neue Dokumentversion angepasst werden müssen,
%    einfach nach 'ZU ERLEDIGEN BEI NEUER DOKUMENTVERSION' suchen. Alle Stellen sind auf diese Weise markiert.


\documentclass[a4paper,10pt,titlepage]{scrartcl}

% META INFO %
\newcommand{\thisdocTITLE}{Ma23 Formelsammlung Dgl'en (gew.)}
\newcommand{\projectURL}{https://github.com/makomi/ma23\_fs/}
% ZU ERLEDIGEN BEI NEUER DOKUMENTVERSION:
% Datum und Version aktualisieren
\newcommand{\thisdocDATE}{2016-01-18 v1.0.3}
% eventuell neuer Herausgeber
\newcommand{\thisdocSUBJECT}{\projectURL} % Herausgeber:
% ZU ERLEDIGEN BEI NEUER DOKUMENTVERSION:
% Neue Autoren anfügen (jeweils durch ein Komma getrennt)
\newcommand{\myNAME}{Matthias Kolja Miehl}
\newcommand{\myEMAIL}{miehl@w3hs.net}
\newcommand{\thisdocAUTHOR}{\myNAME, \myEMAIL}

% linespacing
\usepackage{setspace}
\spacing{0.8}
%\linespread{0.7}

% MISC PACKAGES %
\usepackage[utf8]{inputenc}
\usepackage[ngerman]{babel}
\selectlanguage{ngerman}
\usepackage{vmargin}
\usepackage{lastpage}
\usepackage{url}
% MATH PACKAGES %
\usepackage{amsmath,amssymb,amsthm}


% Document Properties
\usepackage{hyperref}
\hypersetup{%
            pdftitle        ={\thisdocTITLE}
           ,pdfauthor       ={\thisdocAUTHOR}
           ,pdfsubject      ={\thisdocSUBJECT}
           ,pdfkeywords     ={\thisdocDATE}
%           ,pdfcreator      ={}
%
           ,colorlinks      =true                       % false: boxed links; true: colored links
%           ,linkbordercolor ={1 1 1}                    % {r g b} (e.g. white: {1 1 1})
%           ,urlbordercolor  ={1 1 1}                    %
%           ,filebordercolor ={1 1 1}                    %
%           ,citebordercolor ={1 1 1}                    %
           ,linkcolor       =black                      % color of internal links        (e.g. red)
           ,urlcolor        =black                      % color of external links        (e.g. cyan)
           ,filecolor       =black                      % color of file links            (e.g. magenta)
           ,citecolor       =black                      % color of links to bibliography (e.g. green)
%
%           ,pdfnewwindow    =true                       % open links in new window?
%           ,pdfstartview    =FitBH                      % FitBH: fit width of page to the window
%           ,pdffitwindow    =true                       % page fit to window when opened
%           ,pdftoolbar      =true                       % show Acrobat’s toolbar?
%           ,pdfmenubar      =true                       % show Acrobat’s menu?
%           ,bookmarks       =false                      % show bookmarks bar? (SUGI style guide)
%           ,unicode         =true                       % false: non-Latin characters in Acrobat’s bookmarks
}

% Maximum margins for 'HP Deskjet 930c'
\setmargins{11px}{11px}       % linker          & oberer Rand
           {20.2cm}{27.8cm}   % Textbreite      & -höhe
           {14pt}{0cm}        % Kopfzeilenhoehe & -abstand
           {0pt}{0cm}         % \footheight     & Fusszeilenabstand

% Document Header
\usepackage{fancyhdr}
\pagestyle{fancy}
 \lhead{}
 \chead{\begin{scriptsize}{\thepage}\,/\,\pageref{LastPage}\end{scriptsize}}
 \rhead{\begin{scriptsize}\href{\projectURL}{\projectURL} \quad \href{mailto:\myEMAIL}{\myEMAIL} \quad \thisdocDATE\end{scriptsize}}
 \lfoot{}
 \cfoot{}
 \rfoot{}
\renewcommand{\headrulewidth}{0.0pt}
\renewcommand{\footrulewidth}{0.0pt}


% USED LABELs
%  'NOTE', 'FIX', 'HERE'
% USED COMMANDs
%  \ll \gg
%  \pm \rmmath{i}

\begin{document}

\pagenumbering{alph}   % a, b, c, ...

\begin{titlepage}
  \vspace*{\fill}
  \begin{center}
    \huge
    Mathematik 2,\,3\\
    Formelsammlung zu gew.\,Dgl'en\\
    \vspace{1.5cm}
    \large
    \thisdocDATE
  \end{center}
  \vspace*{\fill}
  \begin{center}{\fontsize{9pt}{11pt}\selectfont
    % Creative Commons License Notice
    %
    This work is licensed under a\\[1em]
    \textbf{Creative Commons Attribution-ShareAlike 4.0 International License (CC BY-SA 4.0).\\[1em]}
    % More Information:
    \url{http://creativecommons.org/licenses/by-sa/4.0/}\\
  }
  \vspace*{\fill}
% ZU ERLEDIGEN BEI NEUER DOKUMENTVERSION:
% '-- Autor -- \medskip\\' auskommentieren und die darauf folgende Zeile einkommentieren; für jeden Autor eine eigene Zeile anfügen
%
  -- Autor -- \medskip\\
%  -- Autoren -- \medskip\\
  \myNAME, \href{mailto:\myEMAIL}{\myEMAIL}\smallskip\\
  \url{\projectURL}
% Vorlagen für weitere Zeilen:
%  Autor1 der modifizierten Version \\
%  Autor2 der modifizierten Version \\
%  Autor3 der Version, die auf der modifizierten Version aufbaut\\
%  ...
  \vspace*{1cm}
  \end{center}
\end{titlepage}

\newpage

\pagenumbering{arabic}   % 1, 2, 3, ...
\setcounter{page}{1}


\section*{Lsgsverfahren für einzelne Dgl'en}
\label{sec:lsgsverfahren_fuer_einzelne_dglen}

\subsection*{$n$-fache Integration}
\label{sec:n-fache_integration}

Kl.: \emph{Gewöhnliche Dgl}
\subsubsection*{Allg.\,Lösung}
\begin{tabular}{ll}
 \emph{Anwendung:}
&
 $y^{(n)}=f(x)$ \qquad ($f(x)$ kann auch konstant sein: $f(x)=3$)
\\
 \emph{Lösungsweg:}
&
 \parbox{10cm}{%
  1.: Differentialquotient allein auf eine Seite bringen\\
  2.: $n$-fach integrieren (entsprechend der Ordnung der Dgl)
 }
\\
\end{tabular}
\subsubsection*{Spez.\,Lösung}
\begin{tabular}{ll}
 \emph{Anwendung:}
&
 Bei Vorgabe von Anfangsbedingungen für gew.\,Dgl'en $n$.\,Ordnung
\\
\end{tabular}
\smallskip\\
\begin{tabular}{ll}
 \emph{Lösungsweg:}
& 1. Einsetzen der Anfangsbedingungen in die allg.\,Lsg bzw.\,ihre Ableitungen
\\
& $\to$ spezielle Konstanten ($c_0,\,c_1,\,\dots\,,\,c_n$)
\\
& 2. Einsetzen der spez.\,Konstanten in die allg.\,Lsg.
\\
& $\to$ spezielle Lsg.
\\
\end{tabular}
%\medskip\\
%\begin{tabular}{ll}
%\emph{Bemerkung:} & Einfachstes Lsgsverfahren\\
%\end{tabular}


\subsection*{Trennung der Variablen}
\label{sec:trennung_der_variablen}

Kl: \emph{Gewöhnliche Dgl 1.\,Ordnung}
\subsubsection*{Allg.\,Lösung}
\begin{tabular}{ll}
 \emph{Anwendung:}
&
 $y'=f(x)\cdot g(y)$ \qquad (z.B. $y'=\cos x\cdot y^2$)
\\
\end{tabular}
\smallskip\\
\begin{tabular}{ll}
 \emph{Lösungsweg:}
& 1. Ableitung ausführlich als \emph{Quotient} der Differentiale schreiben
\\
& 2. Versch.\,Variable (u.\,entsprechende Differentiale) auf versch.\,Seiten bringen ($\mathrm{d}x$ und $\mathrm{d}y$ dabei jeweils im Zähler!)
\\
& 3. Integrieren
\\
& 4. Auflösen nach der abhängigen Variablen (falls möglich) (hier: $y$)
\\
\end{tabular}
%\medskip\\
%\begin{tabular}{ll}
% \emph{Bemerkung:}
% FIX: Stimmt die folgende Aussage? (Quelle: Papula)
%& T.\,d.\,V.\,nur möglich für Anfangsbedingungen mit $x$-Wert $\neq0$\\
%\end{tabular}


\subsection*{Variation der Konstanten}
\label{sec:variation_der_konstanten}
Kl: \emph{Inh.\,lin.\,gew.\,Dgl $n$.\,Ordnung mit var.\,oder konst.\,Koeffiz.}

\subsubsection*{Sonderfall: Dgl 1.\,Ordnung (Allg.\,Lösung)}
\begin{tabular}{ll}
 \emph{Anwendung:}
 & $y'+f(x)\cdot y=S(x)$ \qquad (z.B. $y'+x\cdot y=x$; \emph{kein} gemischtes Produkt $yy'$, da \emph{nicht} linear)\\
\end{tabular}
\smallskip\\
\begin{tabular}{ll}
 \emph{Lösungsweg:}
 & 1. Zu lösende Dgl homogenisieren (Stör-Fkt abspalten)\\
 & 2. Lösen des homogenen Teils; allg.\,Lsg einer hom.\,lin.\,Dgl 1.\,Ordnung: $y_h=c_1\cdot e^{-\int f(x)\,\mathrm{d}x}$\\
 & 3. Berücksichtigen der Stör-Fkt durch\\
 & \quad\, a) Variation der allg.\,Konstante in der Lösung des homogenen Teils: $c_1\to c_1(x)$\\
 & \quad\, b) $1$-faches Differenzieren der \emph{allg.\,Lösung des homogenen Teils mit variierter Konstante}\\
 & \quad\, c) Einsetzen der Ergebnisse von a) und b) in die inhomogene Dgl (soweit dort einsetzbar)\\
 & \qquad\;   Kontrollstelle: Hier fallen stetst zwei Summanden weg\\
 & \quad\, d) Bestimmen der \emph{variierten Konstante} durch Integrieren: $c_1(x)=\int \bigl(S(x)\cdot e^{\int f(x)\,\mathrm{d}x}\bigr)\mathrm{d}x+c_2$\\
 & \quad\, e) Einsetzen des für die variierte Konstante gefundenen Wertes in die\\
 & \qquad\;   \emph{allg.\,Lösung des hom.\,Teils mit variierter Konstante} $\to$ $y=\Bigl(\int \bigl(S(x)\cdot e^{\int f(x)\,\mathrm{d}x}\bigr)\mathrm{d}x+c_2\Bigr)\cdot e^{-\int f(x)\,\mathrm{d}x}$\\
\end{tabular}
\medskip\\
\begin{tabular}{ll}
 \emph{Bemerkung:} & Dieses Verfahren berücksichtigt nur die Stör-Fkt.\\
 & Es muss also immer mit einem anderen Verfahren zur Lösung des homogenen Teils gekoppelt werden.\\
\end{tabular}

\subsubsection*{Sonderfall: Dgl 1.\,Ordnung (Allg.\,Lösung) ($f(x)=a=\text{{\normalfont konst.}}$)}
\begin{tabular}{ll}
 \emph{Anwendung:}
 & $y'+a\cdot y=S(x)$ \; mit \; $a\neq0$ \qquad (z.B. $y'+y=x$; \emph{kein} gemischtes Produkt $yy'$, da \emph{nicht} linear)\\
\end{tabular}
\smallskip\\
\begin{tabular}{ll}
 \emph{Lösungsweg:}
 & 1. Abspalten der Stör-Fkt\\
 & 2. Lösen des homogenen Teils; allg.\,Lsg einer hom.\,Dgl 1.\,Ordnung mit $f(x)=a$: $y_h=c_1\cdot e^{-ax}$\\
 & 3. Berücksichtigen der Stör-Fkt wie im \emph{allgemeinen} Fall (s.\,o.) oder durch einen speziellen Störansatz:\\
 & \quad\, a) Geg.\,Stör-Fkt $S(x)$ ausführlich schreiben\\
 & \quad\, b) Prüfen, ob Fall S1 oder S2 vorliegt (s.\,Tabelle weiter hinten)\\
 & \quad\, c) Erstellen des speziellen Störansatzes. Hierbei ist für den Fall S1 folgendes zu beachten:\\
 & \qquad\;   $\gamma\neq -a$: \; $y_s=e^{\gamma x} A_1(x)$    \qquad\hspace{1.4mm} (keine Resonanz)\\
 & \qquad\;   $\gamma=-a$:     \; $y_s=e^{\gamma x} x A_1(x)$ \qquad               ($1$-fache Resonanz)\\
% NOTE: Es kann keine 2-fache Resonanz auftreten, da es sich hier speziell um eine Dgl 1. Ordnung handelt.
 & \quad\, d) $1$-faches Differenzieren des speziellen Störansatzes\\
 & \quad\, e) Einsetzen des speziellen Störansatzes und seiner Ableitungen in die inh.\,Dgl (soweit erforderlich)\\
 & \quad\, f) Ermitteln der speziellen Koeffizienten des speziellen Störansatzes\\ %(Koeffizientenvergleich)
 & \quad\, g) Einsetzen der gefundenen speziellen Koeffizienten in den speziellen Störansatz $\to$ spezielle Lsg $y_s$\\
 & 4. Gesamtlösung bestimmen durch Addieren der Lsg des hom.\,Teils und der gefundenen spez.\,Lsg\\
 & \quad\, $\to$ $y=y_h+y_s$\\
\end{tabular}

\subsubsection*{Dgl $n$.\,Ordnung (Allg.\,Lösung)}
\begin{tabular}{ll}
 \emph{Anwendung:}
 & $f_n(x)\,y^{(n)}+f_{n-1}(x)\,y^{(n-1)}+\cdots+f_2(x)\,y''+f_1(x)\,y'+f_0(x)\,y=S(x)$ \qquad (z.B. $y'''=1$)\\
\end{tabular}
\smallskip\\
\begin{tabular}{ll}
 \emph{Lösungsweg:}
 & 1. Abspalten der Stör-Fkt\\
 & 2. Lösen des homogenen Teils\\
 & 3. Berücksichtigen der Stör-Fkt durch\\
 & \quad\, a) Variation \emph{aller} allg.\,Konstanten in der Lösung des homogenen Teils\\
 & \quad\, b) $n$-faches Differenzieren der \emph{allg.\,Lösung des homogenen Teils mit variierten Konstanten}\\
 & \qquad\;   Durch Einführen zusätzlicher Bedingungen vermeiden, dass \emph{höhere} Ableitungen der var.\,Konst.\,entstehen\\
 & \quad\, c) Einsetzen der Ergebnisse von a) und b) in die inhomogene Dgl (soweit dort einsetzbar)\\
 & \quad\, d) Zusammenstellen der $n$ Gl'en mit den $n$ unbekannten Fkt'en $c'_1(x)$ bis $c'_n(x)$.\\
 & \qquad\;   Es sind die $n-1$ zusätzl.\,Bdgn und die geg.\,Dgl, nachdem die Ableitungen dort eingesetzt worden sind\\
 & \quad\, e) Auflösen der $n$ Gl'en, um $c'_1(x)$ bis $c'_n(x)$ zu erhalten\\
 & \quad\, f) Integrieren von $c'_1(x)$ bis $c'_n(x)$, um $c_1(x)$ bist $c_n(x)$ zu erhalten\\
 & \quad\, g) Einsetzen der für die variierten Konstanten gefundenen Werte in die \emph{allg.\,Lsg d.\,hom.\,Teils mit var.\,Konst.}\\
\end{tabular}
\medskip\\
\begin{tabular}{ll}
 \emph{Bemerkung:} & Ist $S(x)$ bei inh.\,lin.\,gew.\,Dgl'en mit \emph{konst.}\,Koeffiz.\,intervallweise unterschiedlich,\\
 & so bietet sich u.\,a.\,die Lösung mittels Laplace-Transformation an.\\
\end{tabular}


\subsection*{Substitution}
\label{sec:substitution}
Kl: \emph{Gew.\,Dgl $n$.\,Ordnung}

\subsubsection*{Allg.\,Lösung}
\begin{tabular}{ll}
 \emph{Anwendung:} & Viele verschiedene Substitutionen sind möglich. Hier einige Beispiele:\\
\end{tabular}
\begin{center}
\begin{tabular}{l|l|l}
 Gleichung & Substitution(s-Gl) & Neue Dgl\,/\,Lsgsweg \; (statt 2.\,u.\,3.)\\ \hline
 $y'=f\left(\frac{y}{x}\right)$ \quad (homogene Dgl) & $z=\frac{y}{x}$ 
 &
 \parbox{5.1cm}{%
  $z'=\frac{f(z)-z}{x}$\\
  1. T.\,d.\,V. \quad
  2. Rücksubstitution\smallskip\\
 }\\
 $y'=f(ax+by+c)$ & $z=ax+by+c$
 &
 \parbox{5.1cm}{%
  $z'=a+b\cdot f(z)$\\
  1. T.\,d.\,V. \quad
  2. Rücksubstitution\\
 }\\
 $y^{(n)}=f(y^{(m)}; x)$ \;mit\; $n>m$ \quad\; (Reduktion der Ordnung) & $z=y^{(m)}$ & \\
 $y'=y\cdot f(x)+y^n\cdot g(x)$ \; mit \; $n\neq1$ \quad (Bernoulli'sche Dgl) & $z=y^{1-n}$
 &
 \parbox{5.1cm}{%
  $z'=(1-n)\,f(x)\cdot z+(1-n)\,g(x)$\\
  1. lin.\,Dgl \quad
  2. Rücksubstitution
 }\\
\end{tabular}
\end{center}
\begin{tabular}{ll}
 \emph{Lösungsweg:}
 & 1. Substituieren (passende Substitution ermitteln)\\
 & 2. Innerhalb der Substitution:\\
 & \quad\, a) Auflösen der Substitutions-Gl nach der ursprünglichen abhängigen Variablen\\
 & \quad\, b) Ableitung(en) der ursprünglichen abhängigen Variablen bilden (entsprechend der Ordnung der geg.\,Dgl)\\
 & 3. Einsetzen der Subst-Gl u.\,d.\,Abltg(en) der nach der ursprünglichen abh.\,Var.\,aufgelösten Subst.\,in die geg.\,Dgl\\
% NOTE:
% Der folgende auskommentierte Text stammt aus dem Buch an dem ich mich orientiert habe. (Gregor Oberholz - Dgl'en)
% An dieser Stelle stimme ich mit dem Buch nicht überein:
% & 3. Einsetzen der Substitutions-Gl und der Ableitung(en)\\
% & \quad\, (ggf.\,auch der nach der ursprünglichen abhängigen Variablen aufgelösten Substitution) in die geg.\,Dgl\\
 & \quad\, $\to$ Neue Dgl mit der alten unabhängigen und einer neuen abhängigen Variablen\\
 & 4. Lösen dieser neuen Dgl \quad (oft mittels T.\,d.\,V. oder V.\,d.\,K.)\\
 & 5. Rücksubstituieren\\
 & 6. Auflösen nach der abhängigen Variablen (hier: $y$) (falls möglich)\\
% NOTE: '(hier: y)' ist eine Notiz für mich, da ich sonst vllt. y' also dy/dx stehen lassen würde.
\end{tabular}
%\medskip\\
%\begin{tabular}{ll}
% \emph{Bemerkung:} & \\
%\end{tabular}



\subsection*{Nichtlin.\,Dgl'en 1.\,Ordnung}
\label{sec:nicht-lin_dglen_1_ordnung}
Kl: \emph{Inh.\,nichtlin.\,gew.\,Dgl $1$.\,Ordnung mit var.\,oder konst.\,Koeffiz.}

\subsubsection*{Trennung der Variablen}
\begin{tabular}{ll}
 \emph{Anwendung:}
&
 $y'=f(x)\cdot g(y)$ \qquad (z.B. $y'=\cos x\cdot y^2$)
\\
\end{tabular}
\smallskip\\
\begin{tabular}{ll}
 \emph{Lösungsweg:}
& 1. Ableitung ausführlich als \emph{Quotient} der Differentiale schreiben
\\
& 2. Versch.\,Variable (u.\,entsprechende Differentiale) auf versch.\,Seiten bringen ($\mathrm{d}x$ und $\mathrm{d}y$ dabei jeweils im Zähler!)
\\
& 3. Integrieren
\\
& 4. Auflösen nach der abhängigen Variablen (falls möglich) (hier: $y$)
\\
\end{tabular}
%\medskip\\
%\begin{tabular}{ll}
% \emph{Bemerkung:}
% FIX: Stimmt die folgende Aussage? (Quelle: Papula)
% fix: Die zweite Bermerkung zu 'V.d.K.' wieder einkommentieren,
%      wenn sich herausstellen sollte, dass diese Bemerkung zu 'T.d.V.
%      falsch ist, da dann eine weitere Zeile auf dieser Seite zur Verfügung steht.
% & T.\,d.\,V.\,nur möglich für Anfangsbedingungen mit $x$-Wert $\neq0$
% \\
%\end{tabular}

% FIX: Prüfen ob die im Lösungsweg für die V.d.K. unter Punkt d und e angegebenen Gleichungen
%      tatsächlich auch für nichtlineare Dgl'en gelten.
\subsubsection*{Variation der Konstanten}
\begin{tabular}{ll}
 \emph{Anwendung:}
 & $y'+f(x)\cdot g(y)=S(x)$
 \\
\end{tabular}
\smallskip\\
\begin{tabular}{ll}
 \emph{Lösungsweg:}
 & 1. Abspalten der Stör-Fkt\\
 & 2. Lösen des homogenen Teils mit T.\,d.\,V.\\
 & 3. Berücksichtigen der Stör-Fkt durch\\
 & \quad\, a) Variation der allg.\,Konstante in der Lösung des homogenen Teils: $c_1\to c_1(x)$\\
 & \quad\, b) $1$-faches Differenzieren der \emph{allg.\,Lösung des homogenen Teils mit variierter Konstante}\\
 & \quad\, c) Einsetzen der Ergebnisse von a) und b) in die inhomogene Dgl (soweit dort einsetzbar)\\
 & \qquad\;   Kontrollstelle: Hier fallen stetst zwei Summanden weg\\
 & \quad\, d) Bestimmen der \emph{variierten Konstante} durch Integrieren \\%: $c_1(x)=\int \bigl(S(x)\cdot e^{\int f(x)\,\mathrm{d}x}\bigr)\mathrm{d}x+c_2$\\
 & \quad\, e) Einsetzen des für die variierte Konstante gefundenen Wertes in die\\
 & \qquad\;   \emph{allg.\,Lösung des hom.\,Teils mit variierter Konstante} \\% $\to$ $y=\Bigl(\int \bigl(S(x)\cdot e^{\int f(x)\,\mathrm{d}x}\bigr)\mathrm{d}x+c_2\Bigr)\cdot e^{-\int f(x)\,\mathrm{d}x}$\\
\end{tabular}
\medskip\\
\begin{tabular}{ll}
 \emph{Bemerkung:} & Dieses Verfahren berücksichtigt nur die Stör-Fkt.\\
% note: Auskommentiert, da es teilweise außerhalb des von meinem Drucker (HP Deskjet 930C) bedruckbaren Bereichs liegen würde,
%       hier eh nicht so wichtig ist und zudem doppelt wäre.
 & Es muss also immer mit einem anderen Verfahren zur Lösung des homogenen Teils gekoppelt werden.\\
\end{tabular}
\vspace{4mm}
% FIX: Funktioniert das hier auskommentierte Verfahren?
%\bigskip\\
% NOTE: Dies ist die Formel, auf der dieser Abschnitt basiert.
%       Ich habe sie aus einer anderen Formelsammlung, deren Verfasser mir unbekannt ist.
% $y'_{(x)}+a_{(x)}\cdot y_{(x)}=f_{(x)}$ \qquad $\frac{\mathrm{d}}{\mathrm{d}x}\left(\mathrm{e}^{\int a_{(x)}\mathrm{d}x}\cdot y_{(x)}\right)=f_{(x)}\cdot \mathrm{e}^{\int a_{(x)}\mathrm{d}x}$
%\begin{tabular}{ll}
% \emph{Anwendung:}
% & $y'+f(x)\cdot y=S(x)$
% \\
%\end{tabular}
%\smallskip\\
%\begin{tabular}{ll}
% \emph{Lösungsweg:}
% & 1. Folgendermaßen integrieren: $\frac{\mathrm{d}}{\mathrm{d}x}\left(\mathrm{e}^{\int f(x)\,\mathrm{d}x}\cdot y\right)=S(x)\cdot \mathrm{e}^{\int f(x)\,\mathrm{d}x}$\\
% & 2. Nach $y$ auflösen\\
% & 3. Wenn gefordert: Integrationskonstante $c$ durch Einsetzen des Anfangswertes ermitteln\\
%\end{tabular}



\newpage

\subsection*{Spezielle Ansätze}
\label{sec:spezielle_ansaetze}
\subsubsection*{Spez.\,Ansatz, homogene Dgl'en (Allg.\,Lösung)}
Kl: \emph{Homog.\,lin.\,gew.\,Dgl $n$.\,Ordnung mit konst.\,Koeffiz.}
\smallskip\\
\begin{tabular}{ll}
 \emph{Anwendung:} & $y^{(n)}+a_{n-1}\,y^{(n-1)}+\cdots+a_2\,y''+a_1\,y'+a_0\,y=0$ \quad mit \quad $a_0,\,a_1,\,\text{\dots}=$ reell und konstant\\
\end{tabular}
\begin{center}
\begin{tabular}{l|l|l}
 Lsg der char.\,Gl & Allg.\,Lsg der Dgl & Fall\\ \hline
 $k_m=\gamma$ \hspace{1.69cm} reell                        & $y_h=e^{\gamma x} P_1(x)$                                                & C1
 \\
% NOTE @ Fall C1: Spezieller Fall für Fall C2  (omega bzw. Frequenz = 0)
 $k_{m,m+1}=\gamma\pm\mathrm{i}\omega$ \quad konj.-komplex & $y_h=e^{\gamma x}[P_1(x)\cos(\omega x)+P_2(x)\sin(\omega x)]$ & C2
 \\
\end{tabular}
\end{center}
\begin{tabular}{ll}
 \emph{Lösungsweg:}
 & 1. Aufstellen der charakteristischen Gl\\
 & 2. Ermitteln des Falles (C1 oder C2, 1-fach oder mehrfach)\\
 & 3. Einsetzen der Ergebnisse der char.\,Gl in die allg.\,Lsg des entsprechenden Falles\\
 & \quad Grad des Polynoms $P_1(x)$: Anz.\,d.\,\emph{gleichen} Lsg'en der char.\,Gl weniger $1$\\
 & \quad Grade von $P_1(x)$ u.\,$P_2(x)$\,: Anz.\,d.\,\emph{gleichen} konjugiert-komplexen Lsgs\emph{paare} der char.\,Gl weniger $1$\\
 & \quad (Anz.\,d.\,$k$-Werte in der Lsg der char.\,Gl entspricht der Anz.\,d.\,allg.\,Konstanten in der allg.\,Lsg der Dgl)\\
 & 4. Ggf.\,addieren der entsprechenden Teillösungen\\
\end{tabular}


\subsubsection*{Spez.\,Störansatz, inhomogene Dgl'en mit Resonanz (Allg.\,Lösung)}
Kl: \emph{Inhomog.\,lin.\,gew.\,Dgl $n$.\,Ordnung mit konst.\,Koeffiz.}
\smallskip\\
\begin{tabular}{ll}
 \emph{Anwendung:} & $y^{(n)}+a_{n-1}\,y^{(n-1)}+\cdots+a_2\,y''+a_1\,y'+a_0\,y=S(x)$ {\fontsize{9pt}{0pt}\selectfont \; mit \; einigen \emph{spez.}\,Stör-Fkt'en \; und $a_0,\,a_1,\,\text{\dots}=$ reell\,u.\,konst.}\\
\end{tabular}
\begin{center}
\begin{tabular}{l|l|l|l|l}
 Stör-Fkt & Störansatz & Bemerkung & Anwendung & Fall\\ \hline
 $S(x)=e^{\gamma x} B_1(x)$ & $y_s=e^{\gamma x} A_1(x)$ &
 \parbox{3cm}{%
  $\gamma$ ist in $S(x)$ und $y_s$ gleich
 }
 &
 \parbox{6cm}{%
 \emph{spez.\,Fall}:\\
 möglich und günstig nur bei spez.\,$S(x)$ \emph{ohne} trigonom.\,Anteil ($\omega=0$)
% HERE: Abstand zwischen den Zeilen für S1 und S2
 \smallskip\\
 }
 &
 S1
 \\
 \parbox{5cm}{%
 \begin{tabbing}
  $S(x)=$\=$e^{\gamma x} [B_1(x) \cos(\omega x)$ \\
         \>\hspace{3.3mm}$+B_2(x) \sin(\omega x)]$
 \end{tabbing}
 }
 & 
 \parbox{5cm}{%
 \begin{tabbing}
  $y_s=$\=$e^{\gamma x} [A_1(x) \cos(\omega x)$\\
        \>\hspace{3.3mm}$+A_2(x) \sin(\omega x)]$
 \end{tabbing}
 }
 &
 \parbox{3cm}{%
  $\gamma$ und $\omega$ sind in $S(x)$ und $y_s$ gleich
 }
 &
 \parbox{6.5cm}{%
 \emph{allg.\,Fall}:\\
 möglich bei jedem der spez.\,$S(x)$, günstig nur bei spez.\,$S(x)$ \emph{mit} trigonom.\,Anteil
 }
 &
 S2
 \\
\end{tabular}
\end{center}
$B_1(x)$ und $A_1(x)$ sind Polynome in $x$ mit in der Regel \emph{gleichen} maximalen Potenzen (Graden); Ausnahme: Resonanzfall
% FIX: Wann sind die maximalen Potenzen nicht gleich? Meine Vermutung: Bei Resonanz. Ist die Vermutung richtig? Gibt es noch mehr Möglichkeiten?
% NOTE: B(x)  wie  'bestimmtes'  Polynom
%       A(x)  wie  'allgemeines' Polynom
\smallskip\\
Fehlt im Polynom der Stör-Fkt ein Glied niedrigerer Potenz, so muss es im Störansatz trotzdem angesetzt werden:\\
\indent $\to$ \emph{kontinuierlicher Ansatz aller Elemente bis zur höchsten Potenz}
\smallskip\\
$B_1(x)$ und $B_2(x)$ können verschiedene Grade haben. Die Polynome $A_1(x)$ und $A_2(x)$ jedoch müssen untereinander den \emph{gleichen Grad} haben, aber \emph{verschiedene Koeffizienten}. Gewählt wird der \emph{höhere} Grad von $B_1(x)$ und $B_2(x)$.
\smallskip\\
Ist die Stör-Fkt einer Dgl eine Summe aus mehreren verschiedenen Stör-Fkt'en, so muss \emph{jede} Stör-Fkt für sich berücksichtigt werden. Die Gesamtlösung der Dgl ist die Summe aus der Lsg des homogenen Teils und den Lsg'en, die durch die speziellen Störansätze gefunden werden: $y=y_h+y_{S1}+y_{S2}+\cdots+y_{Sn}$\\
Unter dem Begriff \emph{Teil-Dgl} ist folgendes zu verstehen: Dieser Störansatz und seine Ableitungen werden nicht in die \emph{gesamte} vorgegebene inhomogene Dgl eingesetzt, sondern nur in einen \emph{Teil} hiervon. Dieser Teil besteht aus dem homogenen Teil der geg.\,Dgl und der gerade betrachteten Stör-Fkt.
\smallskip\\
%FIX: Wie vorgehen, wenn die Stör-Fkt das PRODUKT aus mehreren verschiedenen Stör-Fkt'en ist? (Hinweis s. Papula Band 2 S.465)
%     S(x) = S_1(x) * S_2(x)  ->   Lösungsansätze y_{p1} und y_{p2} für die beiden Faktorfunktionen S_1(x) und S_2(x) aufschreiben und miteinander multiplizieren  ->  y_p = y_{p1} * y_{p2}
Es liegt \emph{Resonanz} vor, wenn die Lsg des homogenen Teils der Dgl oder Teile davon und die Stör-Fkt gleiches $\gamma$ \emph{und} $\omega$ haben. Treten gleiches $\gamma$ \emph{und} $\omega$ in der Lsg des homogenen Teils bzw.\,der char.\,Gl $n$-fach auf, so liegt $n$-fache Resonanz vor.

\begin{center}
\begin{tabular}{c|c|c|c}
 \multicolumn{2}{l}{Zusammentreffen} \vline & \multicolumn{2}{l}{Resonanz}\\ \hline
 \parbox{2.6cm}{%
 Lsg hom.\,Teil\\
 bzw.\,Lsg char.\,Gl
 }
 & Stör-Fkt & Möglichkeit & Überprüfung\\ \hline
 C1 & S1 &  & $\gamma$\\
 C2 & S2 & \raisebox{1.25ex}[0ex]{ja} & $\gamma$, $\omega$\\ \hline
 C1 & S2 &  &  \\
 C2 & S1 & \raisebox{1.25ex}[0ex]{nein} &  \raisebox{1.25ex}[0ex]{-}\\
\end{tabular}
\end{center}
\begin{tabular}{ll}
 \emph{Lösungsweg:}
 & 1. Abspalten der Stör-Fkt\\
 & 2. Lösen des homogenen Teils $\to$ $y_h$\\
 & 3. Berücksichtigen der Stör-Fkt durch speziellen Störansatz\\
 & \quad\, a) Geg.\,Stör-Fkt $S(x)$ ausführlich schreiben\\
 & \quad\, b) Prüfen, ob Fall S1 oder S2 vorliegt\\
 & \quad\, c) Erstellen des (im Falle der Resonanz vorläufigen) speziellen Störansatzes\\
 & \quad\, d) Prüfen, ob Resonanz möglich (durch Vergleich der Stör-Fkt mit der Lsg des homogenen Teils)\\%, ob Zusammentreffen von C1-S1 oder C2-S2)\\
 & \quad\, e) Falls ja: Prüfen, ob Resonanz vorhanden ist %(Übereinstimmung von $\gamma$ bzw.\,$\gamma$ \emph{und} $\omega$)
              und ggf.\,Störansatz korrigieren\\
 & \qquad\;   (mit Faktor $x^n$ multiplizieren, wobei $n$ angibt, wievielfache Resonanz vorliegt)\\
 & \qquad\;   $\to$ endgültiger spezieller Störansatz\\
 & \quad\, f) Differenzieren des speziellen Störansatzes entsprechend der Ordnung der Dgl\\
 & \quad\, g) Einsetzen des speziellen Störansatzes und seiner Ableitungen in die inh.\,Dgl (soweit erforderlich)\\
 & \quad\, h) Ermitteln der speziellen Koeffizienten des speziellen Störansatzes (Koeffizientenvergleich)\\
 & \quad\, i) \;Einsetzen der gefundenen speziellen Koeffizienten in den speziellen Störansatz $\to$ spezielle Lsg $y_s$\\
 & 4. Gesamtlösung bestimmen durch Addieren der Lsg des hom.\,Teils und der gefundenen spez.\,Lsg\\
 & \quad\, $\to$ $y=y_h+y_s$\\
\end{tabular}
\medskip\\
\begin{tabular}{ll}
 \emph{Bemerkungen:}
 & 1. Nur zur Berücksichtigung der Stör-Fkt\\
 & 2. Lsg des hom.\,Teils u.\,Stör-Fkt möglichst ausführlich schreiben $\to$ Resonanzfälle leichter zu erkennen\\
 & 3. Es brauchen nur die Störansätze von den Stör-Fkt'en korrigiert zu werden, bei denen Resonanz auftritt.\\
% & 3. Hat eine inh.\,Dgl mehrere Stör-Fkt'en und tritt nur bei einigen davon Resonanz auf,\\
% & \quad\, so brauchen nur die entsprechenden Störansätze korrigiert zu werden.\\
 & 4. Alternativlösungsweg: V.\,d.\,K.; hier: Resonanz muss nicht speziell berücksichtigt werden\\
\end{tabular}

\end{document}
