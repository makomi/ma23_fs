%    Copyright (c)  2008, 2010, 2012, 2016  Matthias Kolja Miehl <miehl@w3hs.net>
%    This work is licensed under a Creative Commons Attribution-ShareAlike 4.0 International License (CC BY-SA 4.0).
%     - Attribution
%     - Share Alike
%
%    http://creativecommons.org/licenses/by-sa/4.0/
%
%    This license allows you to remix, tweak, and build upon the material,
%    as long as you credit the copyright holder and license your new
%    creations under the identical terms.


%    Motivation
%    Durch die Veröffenltichung der Formelsammlung unter der 'CC BY-SA 4.0' soll erreicht werden,
%    dass die Quelltexte aller Dokumente, die auf diesem Quelltext basieren, wiederum für jeden zugänglich sind.
%    Somit soll jeder von der Arbeit der anderen profitieren können und selbst die Möglichkeit haben andere
%    von seiner Arbeit profitieren zu lassen.
%    Nichts ist ärgerlicher als eine bereits getane Arbeit nur deshalb nocheinmal erledigen zu müssen,
%    weil man die ursprüngliche nicht für die eigenen Zwecke nutzen (z.B. für sich anpassen) darf.


%    ## THEMATISCHE AUSSPARUNGEN
%     # AUSSAGENLOGIK
%       Neutralität, de Morgan
%     # PRÄDIKATENLOGIK
%       Die meisten Bspe f Identitäten und Implikationen (aus der VL von Prof. Schiffer)
%     # RELATIONEN
%    


%    Tipp zum Erstellen einer neuen Dokumentversion
%    Um die Stellen im Quelltext zu finden, die für eine neue Dokumentversion angepasst werden müssen,
%    einfach nach 'ZU ERLEDIGEN BEI NEUER DOKUMENTVERSION' suchen. Alle Stellen sind auf diese Weise markiert.


\documentclass[a4paper,10pt,titlepage]{scrartcl}

% META INFO %
\newcommand{\thisdocTITLE}{Ma23 Formelsammlung Logik}
\newcommand{\projectURL}{https://github.com/makomi/ma23\_fs/}
% ZU ERLEDIGEN BEI NEUER DOKUMENTVERSION:
% Datum und Version aktualisieren
\newcommand{\thisdocDATE}{2016-01-18 v1.0.3}
% eventuell neuer Herausgeber
\newcommand{\thisdocSUBJECT}{\projectURL} % Herausgeber:
% ZU ERLEDIGEN BEI NEUER DOKUMENTVERSION:
% Neue Autoren anfügen (jeweils durch ein Komma getrennt)
\newcommand{\myNAME}{Matthias Kolja Miehl}
\newcommand{\myEMAIL}{miehl@w3hs.net}
\newcommand{\thisdocAUTHOR}{\myNAME, \myEMAIL}

% linespacing
\usepackage{setspace}
\spacing{0.8}
%\linespread{0.7}

% MISC PACKAGES %
\usepackage[utf8]{inputenc}
\usepackage[ngerman]{babel}
\selectlanguage{ngerman}
\usepackage{vmargin}
\usepackage{url}
% MATH PACKAGES %
\usepackage{amsmath,amssymb,amsthm}


% Document Properties
\usepackage{hyperref}
\hypersetup{%
            pdftitle        ={\thisdocTITLE}
           ,pdfauthor       ={\thisdocAUTHOR}
           ,pdfsubject      ={\thisdocSUBJECT}
           ,pdfkeywords     ={\thisdocDATE}
%           ,pdfcreator      ={}
%
           ,colorlinks      =true                       % false: boxed links; true: colored links
%           ,linkbordercolor ={1 1 1}                    % {r g b} (e.g. white: {1 1 1})
%           ,urlbordercolor  ={1 1 1}                    %
%           ,filebordercolor ={1 1 1}                    %
%           ,citebordercolor ={1 1 1}                    %
           ,linkcolor       =black                      % color of internal links        (e.g. red)
           ,urlcolor        =black                      % color of external links        (e.g. cyan)
           ,filecolor       =black                      % color of file links            (e.g. magenta)
           ,citecolor       =black                      % color of links to bibliography (e.g. green)
%
%           ,pdfnewwindow    =true                       % open links in new window?
%           ,pdfstartview    =FitBH                      % FitBH: fit width of page to the window
%           ,pdffitwindow    =true                       % page fit to window when opened
%           ,pdftoolbar      =true                       % show Acrobat’s toolbar?
%           ,pdfmenubar      =true                       % show Acrobat’s menu?
%           ,bookmarks       =false                      % show bookmarks bar? (SUGI style guide)
%           ,unicode         =true                       % false: non-Latin characters in Acrobat’s bookmarks
}

% Maximum margins for 'HP Deskjet 930c'
\setmargins{11px}{11px}       % linker          & oberer Rand
           {20.2cm}{27.6cm}   % Textbreite      & -höhe
           {14pt}{0cm}        % Kopfzeilenhoehe & -abstand
           {0pt}{0cm}         % \footheight     & Fusszeilenabstand

% Document Header
\usepackage{fancyhdr}
\pagestyle{fancy}
 \lhead{}
 \chead{}
 \rhead{\begin{scriptsize}\href{\projectURL}{\projectURL} \quad \href{mailto:\myEMAIL}{\myEMAIL} \quad \thisdocDATE\end{scriptsize}}
 \lfoot{}
 \cfoot{}
 \rfoot{}
\renewcommand{\headrulewidth}{0.0pt}
\renewcommand{\footrulewidth}{0.0pt}


% USED LABELs
% 
% USED COMMANDs
% \vee \wedge \cup \cap
% \Leftrightarrow \rightarrow
% \neg \forall \exists
% \in \notin \subseteq
% \emptyset
% \mathrel{\widehat{=}}
% \lbrace \rbrace

\begin{document}

\pagenumbering{alph}   % a, b, c, ...

\begin{titlepage}
  \vspace*{\fill}
  \begin{center}
    \huge
    Mathematik 2,\,3\\
    Formelsammlung zur Logik\\
    \vspace{1.5cm}
    \large
    \thisdocDATE
  \end{center}
  \vspace*{\fill}
  \begin{center}{\fontsize{9pt}{11pt}\selectfont
    % Creative Commons License Notice
    %
    This work is licensed under a\\[1em]
    \textbf{Creative Commons Attribution-ShareAlike 4.0 International License (CC BY-SA 4.0).\\[1em]}
    % More Information:
    \url{http://creativecommons.org/licenses/by-sa/4.0/}\\
  }
  \vspace*{\fill}
% ZU ERLEDIGEN BEI NEUER DOKUMENTVERSION:
% '-- Autor -- \medskip\\' auskommentieren und die darauf folgende Zeile einkommentieren; für jeden Autor eine eigene Zeile anfügen
%
  -- Autor -- \medskip\\
%  -- Autoren -- \medskip\\
  \myNAME, \href{mailto:\myEMAIL}{\myEMAIL}\smallskip\\
  \url{\projectURL}
% Vorlagen für weitere Zeilen:
%  Autor1 der modifizierten Version \\
%  Autor2 der modifizierten Version \\
%  Autor3 der Version, die auf der modifizierten Version aufbaut\\
%  ...
  \vspace*{.8cm}
  \end{center}
\end{titlepage}

\newpage

\pagenumbering{arabic}   % 1, 2, 3, ...
\setcounter{page}{1}


\section*{Aussagenlogik}
\label{sec:aussagenlogik}
\begin{tabular}{lll}
  $\neg\text{, }\wedge\text{, }\vee\text{, }\to\text{, }\leftrightarrow$
& \quad $\wedge\;\cap\;\cdot\;\text{Schnittmenge}$
& \quad $\vee\;\cup\;+\;\text{Vereinigungsmenge}$
\\
\end{tabular}

\subsection*{Logische Identitäten}
\label{sec:logische_identitaeten}
\noindent
\begin{tabular}{l}
$(A\to B)\Leftrightarrow (\neg A\vee B)\Leftrightarrow (\neg B \to \neg A)$ \quad Nur bei $1\to0$ Null. Immer bei $0\to\dots$ Eins.\\
$(A\leftrightarrow B)\Leftrightarrow \bigl((A\to B)\wedge (B\to A)\bigr)$\\
\end{tabular}
\medskip\\

\noindent
\begin{tabular}{lll}
  $\bigl((A\vee B)\wedge A\bigr)\Leftrightarrow A$
& $\bigl((\neg A\vee B)\wedge A\bigr) \Leftrightarrow (A\wedge B)$
& $\bigl((A\vee \neg B)\wedge \neg A\bigr) \Leftrightarrow (\neg A\wedge \neg B)$
\\
  $\bigl((A\vee B)\wedge \neg A\bigr) \Leftrightarrow (\neg A\wedge B)$
& $\bigl((\neg A\vee B)\wedge B\bigr) \Leftrightarrow (\neg A\wedge B)$
& 
\\
\end{tabular}
\smallskip\\

\noindent
\begin{tabular}{ll}
  $\bigl(A\wedge(B\vee C)\bigr)\Leftrightarrow \bigl((A\wedge B)\vee(A\wedge C)\bigr)$
& $\bigl((A\vee B\vee C)\wedge (\neg A\vee B\vee C)\bigr) \Leftrightarrow (\neg A\vee B\vee C)$
\\
  $\bigl((A\vee B)\wedge (\neg A\vee B)\bigr) \Leftrightarrow (\neg A\vee B)$
& $\bigl((A\vee \neg B\vee C)\wedge (\neg A\vee B\vee C)\bigr) \Leftrightarrow \bigl((A\vee \neg B)\wedge (A\vee C)\wedge (B\vee \neg A)\wedge (B\vee C)\bigr)$
\\
\end{tabular}
\medskip\\

\noindent
\begin{tabular}{ll}
  Bsp.\,f.\,1.\,Ers-Th.:
& $\bigl((A\wedge B)\vee (C\wedge D)\bigr) \Leftrightarrow \Bigl(\bigl(A\vee (C\wedge D)\bigr)\wedge \bigl(B\vee (C\wedge D)\bigr)\Bigr)\Leftrightarrow \bigl((A\vee C)\wedge (A\vee D)\wedge (B\vee C)\wedge (B\vee D)\bigr)$
\\
  Bsp.\,f.\,2.\,Ers-Th.:
& $\bigl((A\vee B\vee C)\wedge (\neg A\vee C)\bigr) \Leftrightarrow \bigl((\neg A \wedge B)\vee C\bigr)$
\\
  C ausklammern:
& $\bigl((A\vee B\vee C)\wedge (\neg A\vee C)\bigr) \Leftrightarrow \Bigl(C\vee \bigl((A\vee B)\wedge \neg A\bigr)\Bigr)$ \quad {\fontsize{9pt}{0pt}\selectfont Wenn man ausmultipliziert, dann alles ausmultiplizieren!}
\\
\end{tabular}


\subsection*{Sprachgefühl}
\label{sec:sprachgefuehl}
\noindent
\begin{tabular}{ll}
\emph{Wenn} A, \emph{dann} B							& $(A\to B)$\\
\emph{Wenn} A, \emph{dann niemals} B						& $(A\to \neg B)$\\
\emph{Wenn} A, \emph{dann} von B und C nur B					& $\bigl(A\to(B\wedge \neg C)\bigr)$\\
\emph{Nur wenn} A, \emph{dann} B						& $(B\to A)$\\
\emph{Nur dann} ohne A, \emph{wenn} B						& $(\neg A\to B)$\\
\parbox{9cm}{Eigentlich immer B, außer (vllt.) wenn A}				& $(\neg B\to A)$\\
A jedenfalls \emph{dann}, \emph{wenn} von B und C \emph{höchstens eines}	& $\bigl((\neg B\vee\neg C)\to A\bigr)$\\
A \emph{nur dann},\emph{wenn} nicht B und C					& $\bigl(A\to\neg(B\wedge C)\bigr)$\\
\end{tabular}
\smallskip\\
\noindent
\begin{tabular}{ll}
\parbox{9cm}{\emph{Entweder} A \emph{oder} B}					& $\bigl((A\vee B)\wedge(\neg A\vee\neg B)\bigr)$\\
\emph{Entweder} A und B \emph{oder} keins von beidem				& $\bigl((\neg A \vee B)\wedge(A\vee\neg B)\bigr) \Leftrightarrow (A\leftrightarrow B)$\\
\end{tabular}
\smallskip\\
\noindent
\begin{tabular}{ll}
\parbox{9cm}{\emph{Mindestens} A \emph{oder} B}					& $(A\vee B)$\\
Nicht gleichzeitig A und B (höchstens eines von beidem)				& $\neg(A\wedge B)\Leftrightarrow(\neg A\vee\neg B)$\\
\emph{Keins} von beidem								& $(\neg A\wedge\neg B)\Leftrightarrow\neg(A\vee B)$\\
\end{tabular}


\section*{Prädikatenlogik}
\label{sec:praedikatenlogik}
\begin{tabular}{llll}
$\forall\text{, }\exists\text{, }\neg\text{, }\wedge\text{, }\vee\text{, }\to\text{, }\leftrightarrow$
& \quad $\forall x\, \exists y\, P(x,y)\nLeftrightarrow \exists y\, \forall x\, P(x,y)$
& \qquad \qquad $\exists x\,\bigl(S(x)\wedge I(x)\bigr)$
& \quad $\forall x\, \bigl(S(x)\to I(x)\bigr)$\\
\end{tabular}
\smallskip\\
\noindent
\begin{tabular}{lll}
 \parbox{3.6cm}{%
  $\neg\exists x\, P(x) \Leftrightarrow \forall x\, \neg P(x)$\\
  $\neg\forall x\, P(x) \Leftrightarrow \exists x\, \neg P(x)$
 }
&
 \parbox{6.2cm}{%
  $\forall x\, P(x) \wedge \forall x\, Q(x) \Leftrightarrow \forall x\, \bigl(P(x)\wedge Q(x)\bigr)$\\
  $\exists x\, P(x) \vee   \exists x\, Q(x) \Leftrightarrow \exists x\, \bigl(P(x)\vee   Q(x)\bigr)$
 }
&
 \parbox{2cm}{%
  ${\exists x\,\bigl(P(x)\wedge Q(x)\bigr) \Rightarrow \exists x\, P(x) \wedge \exists x\, Q(x)}$\\
  ${\forall x\, P(x) \vee \forall x\, Q(x) \Rightarrow \forall x\, \bigl(P(x)\vee Q(x)\bigr)}$
 }
\\
\end{tabular}



\section*{Binäre Relationen {\fontsize{9pt}{0pt}\selectfont $(R\subseteq X\times Y)$}}
\label{sec:binaere_relationen}
Gegenbeispiele finden, da bereits ein Gegenbeispiel zur Widerlegung genügt!

\subsection*{Spezielle Eigenschaften}
\label{sec:spezielle_eigenschaften}
\begin{tabular}{lll}
  linkstotal
& {\fontsize{8pt}{0pt}\selectfont $\forall x\, \exists y:(x,y)\in R$}
& {\fontsize{8pt}{0pt}\selectfont Für y erlaubte Zahlen aus Zahlenmenge einsetzen und prüfen, ob alle möglichen x-Werte erreicht werden können.}
\\
  surjektiv
& {\fontsize{8pt}{0pt}\selectfont $\forall y\, \exists x : (x,y)\in R$}
& {\fontsize{8pt}{0pt}\selectfont Für x erlaubte Zahlen aus Zahlenmenge einsetzen und prüfen, ob alle möglichen y-Werte erreicht werden können.}
\\
\end{tabular}
\smallskip\\
\begin{tabular}{lll}
 \parbox{2.2cm}{%
  {\fontsize{10pt}{0pt}\selectfont injektiv}\\
  {\fontsize{10pt}{0pt}\selectfont rechtseindeutig}
 }
&
 \parbox{6.7cm}{%
  {\fontsize{8pt}{0pt}\selectfont $\forall x_1\, \forall x_2\, \forall y   : \bigl((x_1,y)\in R\wedge(x_2,y)\in R\Rightarrow(x_1 = x_2)\bigr)$}\\
  {\fontsize{8pt}{0pt}\selectfont $\forall x\,   \forall y_1\, \forall y_2 : \bigl((x,y_1)\in R\wedge(x,y_2)\in R\Rightarrow(y_1 = y_2)\bigr)$}
 }
&
 \parbox{10cm}{%
  {\fontsize{8pt}{0pt}\selectfont Jedem Element $\in$ Y genau/höchstens ein Element $\in$ X zugeordnet.}\\
  {\fontsize{8pt}{0pt}\selectfont Jedem Element $\in$ X genau/höchstens ein Element $\in$ Y zugeordnet.}
 }
\\
\end{tabular}

\subsection*{Spezielle Typen}
\label{sec:spezielle_typen}
\begin{tabular}{lll}
  reflexiv
& $\forall x\,\bigl((x,x)\in R\bigr)$ \; $I_A\subseteq R$
& Elemente auf HD sind alles Einsen. \emph{Jedes} Element steht mit sich selbst in Relation.
\\
  irreflexiv
& $\forall x\,\bigl((x,x)\notin R\bigr)$ \; $R\cap I_A=\emptyset$
& Elemente auf HD sind alles Nullen. \emph{Kein} Element steht mit sich selbst in Relation.
\\
 
&
& irreflexiv, wenn: asymmetrisch
\\
\end{tabular}
\smallskip\\
{\fontsize{8pt}{0pt}\selectfont
\begin{tabular}{lll}
symmetrisch
& $\forall x\, \forall y\, \bigl((x,y)\in R\Rightarrow(y,x)\in R\bigr)$ \; $R=R^{-1}$
& HD beliebig. Für \emph{jedes} Element $(x,y)\in R$ gibt es auch ein Element $(y,x)\in R$.
\\
asymmetrisch
& $\forall x\, \forall y\, \bigl((x,y)\in R\Rightarrow(y,x)\notin R\bigr)$ \; $R\cap R^{-1}=\emptyset$
& HD nur Nullen. \emph{Kein} Element $(x,y)\in R$ hat ein zugehöriges Element $(y,x)\in R$.
\\
 & (irreflexiv und antisymmetrisch) & \emph{nicht} asymmetrisch, wenn: symmetrisch (außer bei $\emptyset$), nicht irreflexiv bzw.\,reflexiv
\\
\end{tabular}
\\
\begin{tabular}{lll}
  antisymmetrisch
& $\forall x\, \forall y\, \bigl((x,y)\in R\wedge(y,x)\in R\Rightarrow (x=y)\bigr)$ \; $R\cap R^{-1}\subseteq I_A$
& HD beliebig. Gegenbsp.: $\exists x\, \exists y\, \bigl((x,y)\in R\wedge(y,x)\in R\wedge(x\neq y)\bigr)$
\\
 
&
& antisymmetrisch, wenn: asymmetrisch
\\
\end{tabular}
\smallskip\\
\begin{tabular}{lll}
transitiv
& $\forall x\, \forall y\, \forall z\, \bigl((x,y)\in R\wedge(y,z)\in R\Rightarrow(x,z)\in R\bigr)$ \; $R\circ R\subseteq R$
& {\fontsize{9pt}{0pt}\selectfont Gegenbsp.: $\exists x\, \exists y\, \exists z\, \bigl((x,y)\in R\wedge(y,z)\in R\wedge(x,z)\notin R\bigr)$}
\\
 
&
& Für jede Eins in $M_{R\circ R}$ muss an der entspr.\,Stelle in $M_R$ auch eine Eins sein.
\\
\end{tabular}
}
\smallskip\\
{\fontsize{9pt}{0pt}\selectfont
Es gibt Relationen, die weder reflexiv noch irreflexiv sind, aber keine, die beides sind.\\
Es gibt Relationen, die weder symmetrisch noch asymmetrisch sind -- $\emptyset$ ist beides.\\
Eine Relation kann sowohl symmetrisch als auch antisymmetrisch sein.
\smallskip\\
\noindent
Totale Fkt.: linkstotal, rechtseindeutig \qquad Partielle Fkt.: \emph{nicht} linkstotal, rechtseindeutig \qquad \{Def.-Bereich\}\,$\to$\,\{Wertebereich\}\\
Äquivalenzrelation: symmetrisch, reflexiv, transitiv ($R[a]=\lbrace x\in A : (a,x)\in R \rbrace$ {\fontsize{8pt}{0pt}\selectfont Menge aller Elemente $\in$ A mit denen $a$ in Relation steht.})\\
Ordnungsrelation: antisymmetrisch, reflexiv, transitiv $\bigl(x\leq_R y \mathrel{\widehat{=}} (x,y)\in R\bigr)$
}
\end{document}
